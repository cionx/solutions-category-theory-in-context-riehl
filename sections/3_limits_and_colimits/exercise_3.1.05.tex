\subsection{}

The general definition of a cone over~$F$ is as follows:
\begin{quote}
	\itshape
	A cone over~$F$ with summit~$p$, where~$p$ is some object in~$\cat{P}$, is a family~$(λ_j)_{j ∈ \cat{J}}$ of morphisms~$λ_j \colon p \to F j$ with~$j ∈ \cat{J}$, subject to the commutivity of the triangular diagram
	\[
		\begin{tikzcd}[column sep = normal]
			{}
			&
			p
			\arrow{dl}[above left]{λ_j}
			\arrow{dr}[above right]{λ_k}
			&
			{}
			\\
			F j
			\arrow{rr}[above]{F u}
			&
			{}
			&
			F k
		\end{tikzcd}
	\]
	for every morphism~$u \colon j \to k$ in~$\cat{J}$.
\end{quote}

The category~$\cat{P}$ is a poset, whence every diagram in~$\cat{P}$ commutes.
We can therefore simplify the above definition:
\begin{quote}
	\itshape
	A cone over~$F$ with summit~$p$, where~$p$ is some object in~$\cat{P}$, is a family~$(λ_j)_{j ∈ \cat{J}}$ of morphisms~$λ_j \colon p \to F j$ with~$j ∈ \cat{J}$.
\end{quote}

There exists for every object~$p$ of~$\cat{P}$ at most one morphism from~$p$ to~$F j$ in~$\cat{P}$, and this morphism exists if and only if~$p ≤ F j$.
We therefore find the following:
\begin{quote}
	\itshape
	Let~$p$ be an object of~$\cat{P}$.
	There exists a cone on~$F$ with summit~$p$ if and only if~$p ≤ F j$ for every~$j ∈ \cat{J}$.
	This cone is then unique.
\end{quote}

Suppose now that~$p$ and~$p'$ are two summits of cones over~$F$.
Every morphism from~$p$ to~$p'$ is then automatically a morphism of cones, because every diagram in~$\cat{P}$ commutes.
Consequently, there exists a morphism of cones from~$p$ to~$p'$ if and only if there exists a morphism from~$p$ to~$p'$ in~$\cat{P}$, which is the case if and only if~$p ≤ p'$.
We thus find the following:
\begin{quote}
	\itshape
	Let~$p$ be an object of~$\cat{P}$ with~$p ≤ F j$ for every~$j ∈ \cat{J}$, i.e., the summit of a cone over~$F$.
	The cone determined by~$p$ is a limit cone for~$F$ if and only if for every other summit~$p'$ we have~$p' ≤ p$.
\end{quote}

In other words, a limit cone over~$F$ is uniquely determined by its summit, which is precisely a greatest upper bound for the objects~$F j$ with~$j ∈ \cat{J}$.
The limit of~$F$ is thus the infimum of the elements~$F j$ with~$j ∈ \cat{J}$.

Dually, the colimit of~$F$ is the supremum of the elements~$F j$ with~$j ∈ \cat{J}$.
