\subsection{}

Given an object~$c$ of~$\cat{C}$, an element of the set~$\Cone(c, F)$ is a family~$(λ_j)_{j ∈ \cat{J}}$ of morphisms~$λ_j \colon c \to F j$, subject to the commutativity of the triangular diagram
\[
	\begin{tikzcd}[column sep = normal]
		{}
		&
		c
		\arrow{dl}[above left]{λ_j}
		\arrow{dr}[above right]{λ_k}
		&
		{}
		\\
		F j
		\arrow{rr}[above]{F u}
		&
		{}
		&
		F k
	\end{tikzcd}
\]
for every morphism~$u \colon j \to k$ in~$\cat{J}$.

An element~$α$ of the set~$\Hom(Δ(c), F)$ is a natural transformation from~$Δ(c)$ to~$F$.
More explicitly,~$α = (α_j)_{j ∈ J}$ is a family of morphisms~$α_j \colon Δ(c)(j) \to F j$ such that the square diagram
\[
	\begin{tikzcd}
		Δ(c)(j)
		\arrow{r}[above]{Δ(c)(u)}
		\arrow{d}[left]{α_j}
		&
		Δ(c)(k)
		\arrow{d}[right]{α_k}
		\\
		F j
		\arrow{r}[above]{F u}
		&
		F k
	\end{tikzcd}
\]
commutes for every morphism~$u \colon j \to k$ in the index category~$J$.
But we know that~$Δ(c)(j) = c$ for every object~$j$ of~$\cat{J}$, and that~$Δ(c)(u) = \id_c$ for every morphism~$u$ in~$J$.
The above square diagram can therefore be simplified as follows:
\[
	\begin{tikzcd}
		c
		\arrow{r}[above]{\id_c}
		\arrow{d}[left]{α_j}
		&
		c
		\arrow{d}[right]{α_k}
		\\
		F j
		\arrow{r}[above]{F u}
		&
		F k
	\end{tikzcd}
\]
This square diagram commutes if and only if the following triangular diagram commutes:
\[
	\begin{tikzcd}[column sep = normal]
		{}
		&
		c
		\arrow{dl}[above left]{α_j}
		\arrow{dr}[above right]{α_k}
		&
		{}
		\\
		F j
		\arrow{rr}[above]{F u}
		&
		{}
		&
		F k
	\end{tikzcd}
\]

We find that~$α$ the family~$α$ is a natural transformation from~$Δ(c)$ to~$F$ if and only if it is a cone on~$F$ with summit~$c$.
In other words, we have an equality of sets
\begin{equation}
	\label{cones are equal to natural transformations}
	\Hom(Δ(c), F) = \Cone(c, F) \,.%
	\footnote{
		This shows that the two definitions of~$\Cone(c, F)$ provided in Definition~3.1.2 coincide.
	}
\end{equation}

It remains to check that the equality~\eqref{cones are equal to natural transformations} is natural in~$c$.
In other words, we need to check that for every morphism~$f \colon c \to d$ in~$\cat{C}$ the following square diagram commutes:
\[
	\begin{tikzcd}
		\Cone(d, F)
		\arrow{r}[above]{f^*}
		\arrow[equal]{d}
		&
		\Cone(c, F)
		\arrow[equal]{d}
		\\
		\Hom(Δ(d), F)
		\arrow{r}[above]{Δ(f)^*}
		&
		\Hom(Δ(c), F)
	\end{tikzcd}
\]
\begin{itemize*}

	\item
		The map~$f^*$ is given on every cone~$(λ_j)_j$ by
		\[
			f^*( (λ_j)_j ) = (λ_j f)_j \,.
		\]

	\item
		The map~$Δ(f)^*$ is given on every natural transformation~$α = (α_j)_j$ by the components
		\[
			Δ(f)^*(α)_j
			=
			(α ⋅ Δ(f))_j
			=
			α_j Δ(f)_j
			=
			α_j f \,,
		\]
		so that in total
		\[
			Δ(f)^*( (α_j)_j ) = (α_j f)_j \,.
		\]

\end{itemize*}
We see that both maps coincide, as required.
