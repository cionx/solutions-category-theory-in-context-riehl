\subsection{}



\subsubsection{}

We have in~$\cat{C}$ the colimit cone~$κ \colon K \To \colim K$ under~$K$.
Applying the functor~$F$ to the diagram~$K$ and the cone~$κ \colon K \to \colim K$ under it yields the cone
\[
	F κ \colon F K \To F \colim K
\]
under~$F K$ in~$\cat{D}$.
The colimit cone~$γ \colon F K \To \colim F K$ is the initial cone under~$F K$, whence there exists a unique morphism of cones~$f$ as depicted:
\[
	\begin{tikzcd}[column sep = small]
		{}
		&
		F K
		\arrow[Rightarrow]{dl}[above left]{γ}
		\arrow[Rightarrow]{dr}[above right]{F κ}
		&
		{}
		\\
		\colim F K
		\arrow{rr}[above]{f}
		&
		{}
		&
		F \colim K
	\end{tikzcd}
\]



\subsubsection{}

Suppose first that~$F$ preserves colimits.
The cone~$F κ \colon F K \To F \colim K$ is then a colimit cone under~$F K$.
The morphism~$f$ is thus an isomorphism of cones by the uniqueness of colimits.

Suppose now that the morphism~$f$ is an isomorphism in~$\cat{D}$.
It is then an isomorphism of cones from~$γ \colon F K \To \colim F K$ to~$F κ \colon F K \To F \colim K$.
As~$γ \colon F K \To \colim F K$ is initial in the category of cones under~$F K$, it follows that the isomorphic cone~$F κ \colon F K \To F \colim K$ is also initial.
In other words, it is again a colimit cone.
This shows that~$F$ preserves colimits.
