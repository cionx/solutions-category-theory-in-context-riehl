\subsection{}



\subsubsection*{Cones over~$F$}

The category~$\Cones(F)$ of cones over~$F$ is defined as a category of elements of the contravariant functor~$\Cone(\ph, F)$.
This category therefore looks as follows:
\begin{itemize*}

	\item
		The objects of~$\Cones(F)$ are pairs~$(c, (λ_j)_j)$ consisting of an object~$c$ of~$\cat{C}$ and a cone~$(λ_j)_j$ over~$F$ with summit~$c$.
		This means that each~$λ_j$ is a morphism from~$c$ to~$F j$, subject to the commutativity of the triangular diagram
		\[
			\begin{tikzcd}[column sep = normal]
				{}
				&
				c
				\arrow{dl}[above left]{λ_j}
				\arrow{dr}[above right]{λ_k}
				&
				{}
				\\
				F j
				\arrow{rr}[above]{F u}
				&
				{}
				&
				F k
			\end{tikzcd}
		\]
		for every morphism~$u \colon j \to k$ in~$\cat{J}$.

	\item
		A morphism in~$\Cones(F)$ from a cone~$(c, (λ_j)_j)$ to a cone~$(d, (μ_j)_j)$ is a morphism~$f \colon c \to d$ in~$\cat{C}$ such that the triangular diagram
		\[
			\begin{tikzcd}[column sep = normal]
				c
				\arrow{rr}[above]{f}
				\arrow{dr}[below left]{λ_j}
				&
				{}
				&
				d
				\arrow{dl}[below right]{μ_j}
				\\
				{}
				&
				F j
				&
				{}
			\end{tikzcd}
		\]
		commutes for every object~$j$ of~$\cat{J}$.
		In other words, we need to have for the induced map
		\[
			f^*
			\colon
			\Cone(d, F) \to \Cone(c, F) \,,
			\quad
			(μ_j)_j \mapsto (μ_j f)_j
		\]
		the equality~$λ = f^*(μ)$.

\end{itemize*}

The category~$Δ ↓ F$, on the other hand, looks as follows:
\begin{itemize*}

	\item
		The objects of~$Δ ↓ F$ are pairs~$(c, α)$ consisting of an object~$c$ of~$\cat{C}$ and a natural transformation~$α \colon Δ(c) \To F$.

	\item
		A morphism in~$Δ ↓ F$ from an object~$(c, α)$ to an object~$(d, β)$ is a morphism~$f \colon c \to d$ in~$\cat{C}$ such that the following triangular diagram commutes:
		\[
			\begin{tikzcd}[column sep = normal]
				Δ(c)
				\arrow{rr}[above]{Δ(f)}
				\arrow{dr}[below left]{α}
				&
				{}
				&
				Δ(d)
				\arrow{dl}[below right]{β}
				\\
				{}
				&
				F
				&
				{}
			\end{tikzcd}
		\]
		That is, we need to have the equality~$α = Δ(f)^*(β)$.

\end{itemize*}
We have already seen in our solution to the previous exercise (Exercise~3.1.ii) that objects of~$\Cones(F)$ are the same as objects of~$Δ ↓ F$, as a family~$λ = (λ_j)_j$ is a cone over~$F$ with summit~$c$ if and only if~$λ$ is a natural transformation from~$Δ(c)$ to~$F$.

Let~$(c, λ)$ and~$(d, μ)$ be two objects of the two categories, and let~$f \colon c \to d$ be a morphism in~$\cat{C}$.
We have also seen in our solution to the previous exercise that the two induced maps
\[
	f^* \colon \Cone(d, F) \to \Cone(c, F)
\]
and
\[
	Δ(f)^* \colon \Hom(Δ(d), F) \to \Hom(Δ(c), F)
\]
are equal.
We have consequently the sequence of equivalences
\begin{align*}
	{}&
	\text{$f$ is a morphism from~$(c, λ)$ to~$(d, μ)$ in~$\Cones(F)$}
	\\
	\iff{}&
	λ = f^*(μ)
	\\
	\iff{}&
	λ = Δ(f)^*(μ)
	\\
	\iff{}&
	\text{$f$ is a morphism from~$(c, λ)$ to~$(d, μ)$ in~$Δ ↓ F$} \,.
\end{align*}
This shows that not only are the objects of~$\Cones(F)$ and~$Δ ↓ F$ equal, but also their morphisms.

Consequently, the categories~$\Cones(F)$ and~$Δ ↓ F$ are equal.



\subsubsection*{Cocones under~$F$}

We can show in the same way that the category of cocones under~$F$ is equal to the comma category~$F ↓ Δ$.
