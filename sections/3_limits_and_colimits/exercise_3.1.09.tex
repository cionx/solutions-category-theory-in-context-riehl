\subsection{}



\subsubsection*{Cones and initial objects}

Let~$i$ be the initial object of~$\cat{J}$.
There exists for every object~$j$ of~$\cat{C}$ a unique morphism~$u_j$ from~$i$ to~$j$ in~$\cat{J}$.
Let~$ι_j ≔ F u_j$ for every~$j ∈ \cat{J}$, which is a morphism in~$\cat{C}$ from~$F i$ to~$F j$.
We claim that~$ι$ is a limit cone over~$F$ with summit~$F i$.

We first need to check that~$ι$ is a cone over~$F$ with summit~$F i$.
To this end, we need to check that for every morphism~$v \colon j \to k$ in~$\cat{J}$ the following triangular diagram commutes:
\begin{equation}
	\label{original diagram for initial limit cone}
	\begin{tikzcd}[column sep = normal]
		{}
		&
		F i
		\arrow{dl}[above left]{ι_j}
		\arrow{dr}[above right]{ι_k}
		&
		{}
		\\
		F j
		\arrow{rr}[above]{F v}
		&
		{}
		&
		F k
	\end{tikzcd}
\end{equation}
Given that~$ι_j = F u_j$ and~$ι_k = F u_k$, this diagram is the image of the diagram
\[
	\begin{tikzcd}[column sep = normal]
		{}
		&
		i
		\arrow{dl}[above left]{u_j}
		\arrow{dr}[above right]{u_k}
		&
		{}
		\\
		j
		\arrow{rr}[above]{v}
		&
		{}
		&
		k
	\end{tikzcd}
\]
under the functor~$F$.
This original diagram in~$\cat{J}$ commutes because there exists \emph{exactly one} morphism from~$i$ to~$k$ in~$\cat{J}$.
It follows from the functoriality of~$F$ that the original diagram~\eqref{original diagram for initial limit cone} commutes.
This shows that~$ι$ is a cone over~$F$ with summit~$F i$.

We now need to check that~$ι$ is the universal cone over~$F$.
We need to show that for every other cone~$λ$ over~$F$, with cone~$c$, there exists a unique morphism~$f$ from~$c$ to~$F i$ in~$\cat{C}$ that is a morphism of cones from~$λ$ to~$ι$.
\begin{itemize*}

	\item
		We start with the uniqueness.
		Let~$f$ be a morphism of cones from~$λ$ to~$ι$, i.e., a morphism from~$c$ to~$F i$ for which the triangular diagram
		\[
			\begin{tikzcd}[column sep = normal]
				c
				\arrow{rr}[above]{f}
				\arrow{dr}[below left]{λ_j}
				&
				{}
				&
				F i
				\arrow{dl}[below right]{ι_j}
				\\
				{}
				&
				F j
				&
				{}
			\end{tikzcd}
		\]
		commutes for every object~$j$ of~$\cat{J}$.
		We consider the case~$j = i$.
		The unique morphism~$u_i$ from~$i$ to~$i$ in~$\cat{J}$ is necessarily the identity morphism~$\id_i$.
		Consequently,
		\[
			ι_i = F u_i = F \id_i = \id_{F i} \,,
		\]
		and thus
		\[
			f = \id_{F i} f = ι_i f = λ_i \,.
		\]

	\item
		For the existence, we now need to check that the morphism~$λ_i$, which goes from~$c$ to~$F i$, is a morphism of cones from~$λ$ to~$ι$.
		To this end, we need to check that the diagram
		\[
			\begin{tikzcd}[column sep = normal]
				c
				\arrow{rr}[above]{λ_i}
				\arrow{dr}[below left]{λ_j}
				&
				{}
				&
				F i
				\arrow{dl}[below right]{ι_j}
				\\
				{}
				&
				F j
				&
				{}
			\end{tikzcd}
		\]
		commutes for every~$j ∈ \cat{J}$.
		This diagram can be rearranged as follows:
		\[
			\begin{tikzcd}[column sep = normal]
				{}
				&
				c
				\arrow{dl}[above left]{λ_i}
				\arrow{dr}[above right]{λ_j}
				&
				{}
				\\
				F i
				\arrow{rr}[above]{F u_j}
				&
				{}
				&
				F j
			\end{tikzcd}
		\]
		The commutativity of this diagram follows from~$λ$ being a cone over~$F$.

\end{itemize*}

We have thus proven the following:
\begin{quote}
	\itshape
	Let~$\cat{J}$ be a category with initial object~$i$, let~$\cat{C}$ be another category and let~$F \colon \cat{J} \to \cat{C}$ be a diagram of shape~$\cat{J}$ in~$\cat{C}$.
	For every object~$j$ of~$\cat{J}$ let~$u_j$ be the unique morphism from~$i$ to~$j$.
	The family~$(F u_j)_{j ∈ \cat{J}}$ is a limit cone over~$F$ with summit~$F i$.
\end{quote}


\subsubsection*{Cocones and terminal objects}

We can dualize the above result:

\begin{quote}
	\itshape
	Let~$\cat{J}$ be a category with terminal object~$t$, let~$\cat{C}$ be another category and let~$F \colon \cat{J} \to \cat{C}$ be a diagram of shape~$\cat{J}$ in~$\cat{C}$.
	For every object~$j$ of~$\cat{J}$ let~$v_j$ be the unique morphism from~$j$ to~$t$.
	The family~$(F v_j)_{j ∈ \cat{J}}$ is a colimit cone under~$F$ with nadir~$F t$.
\end{quote}

Let now~$α$ be a successor ordinal.
This means that~$α$ is the successor of some other ordinal~$β$.
The ordinal~$β$ is an element of~$α$, and in fact is the unique terminal object of the corresponding category~$\bbgreek{α}$.
Given a diagram~$F$ in a category~$\cat{C}$ of shape~$\bbgreek{α}$, its colimit is thus given by~$F β$.
