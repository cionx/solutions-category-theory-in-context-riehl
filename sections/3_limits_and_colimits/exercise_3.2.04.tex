\subsection{}

A natural transformation~$α \colon F \To G$ is a family~$α = (α_j)_{j ∈ \cat{J}}$ such that for every morphism~$f \colon j \to k$ in~$J$ the following diagram commutes:
\[
	\begin{tikzcd}
		F j
		\arrow{r}[above]{F f}
		\arrow{d}[left]{α_j}
		&
		F k
		\arrow{d}[right]{α_k}
		\\
		G j
		\arrow{r}[above]{G f}
		&
		G k
	\end{tikzcd}
\]
In other words, we need the equality
\[
	α_k ⋅ F f = G f ⋅ α_j
\]
for every morphism~$f \colon j \to k$ in~$\cat{C}$.
We can therefore describe~$\Hom(F, G)$ as the equalizer
\[
	\begin{tikzcd}
		\Hom(F, G)
		\arrow[tail]{r}
		&
		\displaystyle
		∏_{j ∈ \cat{J}} \cat{C}(F j, G j)
		\arrow[shift left]{r}[above]{φ}
		\arrow[shift right]{r}[below]{ψ}
		&
		\displaystyle
		∏_{\substack{f \colon j \to k \\ \text{in~$\cat{J}$}}} \cat{C}(F j, G k)
	\end{tikzcd}
\]
where the two maps~$φ$ and~$ψ$ are given by
\[
	φ( α )_f = α_k ⋅ F f \,,
	\quad
	ψ( α )_f = G f ⋅ α_j \,,
\]
for every morphism~$f \colon j \to k$ in~$\cat{C}$.
