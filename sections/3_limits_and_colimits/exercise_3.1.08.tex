\subsection{}

We label the objects and morphisms in the given diagram as follows:
\[
	\begin{tikzcd}[sep = huge]
		c
		\arrow{r}[above]{f}
		\arrow{d}[left]{h}
		&
		c'
		\arrow{r}[above]{f'}
		\arrow{d}[left]{h'}
		\arrow[phantom]{dr}[description, pos = 0.2]{\ulcorner}
		&
		c''
		\arrow{d}[left]{h''}
		\\
		d
		\arrow{r}[above]{g}
		&
		d'
		\arrow{r}[above]{g'}
		&
		d''
	\end{tikzcd}
\]
We provide a diagram to keep track of the auxiliary morphisms that will be introduced in our argumentations:
\[
	\begin{tikzcd}[sep = huge]
		x
		\arrow[dashed]{dr}[above right]{k}
		\arrow[dashed, bend right]{ddr}[below left]{l}
		\arrow[dashed, bend left = 20]{drr}[below left]{m}
		\arrow[dashed, bend right = 15, crossing over]{ddrr}[pos = 0.8, above right]{p}
		\arrow[dashed, bend left = 25]{drrr}[above right]{q}
		&
		{}
		&
		{}
		&
		{}
		\\
		{}
		&
		c
		\arrow{r}[above, pos = 0.4]{f}
		\arrow{d}[left, near end]{h}
		&
		c'
		\arrow{r}[above]{f'}
		\arrow{d}[left]{h'}
		\arrow[phantom]{dr}[description, pos = 0.2]{\ulcorner}
		&
		c''
		\arrow{d}[left]{h''}
		\\
		{}
		&
		d
		\arrow{r}[below]{g}
		&
		d'
		\arrow{r}[below]{g'}
		&
		d''
	\end{tikzcd}
\]

Suppose first that the left-hand square diagram is a pullback square.
We then have for every object~$x$ of~$\cat{C}$ the sequence of bijections
\begin{align*}
	{}&
	\{ k \suchthat k \colon x \textto c \}
	\\
	≅{}&
	\left\{
		(l, m)
		\suchthat*
		\begin{tabular}{l}
			$l \colon x \to d$ and~$m \colon x \to c'$ \\
			with~$g l = h' m$
		\end{tabular}
	\right\}
	\\
	≅{}&
	\left\{
		(l, p, q)
		\suchthat*
		\begin{tabular}{l}
			$l \colon x \to d$,~$p \colon x \to d'$ and~$q \colon x \to c''$ \\
			with~$g' p = h'' q$ and~$g l = p$
		\end{tabular}
	\right\}
	\\
	≅{}&
	\left\{
		(l, q)
		\suchthat*
		\begin{tabular}{l}
			$l \colon x \to d$ and~$q \colon x \to c''$ \\
			with~$g' g l = h'' q$
		\end{tabular}
	\right\}
\end{align*}
given by
\[
	k
	\mapsto
	(h k, f k)
	\mapsto
	(h k, h' f k, f' f k)
	\mapsto
	(h k, f' f k) \,.
\]
This bijection tells us that the outer rectangular diagram is a pullback square.

Suppose now conversely that the outer rectangular diagram is a pullback square.
We then have for every object~$x$ of~$\cat{C}$ the sequence of bijections
\begin{align*}
	{}&
	\{ k \suchthat k \colon x \textto c \}
	\\
	≅{}&
	\left\{
		(l, q)
		\suchthat*
		\begin{tabular}{l}
			$l \colon x \to d$ and~$q \colon x \to c''$ \\
			with~$g' g l = h'' q$
		\end{tabular}
	\right\}
	\\
	≅{}&
	\left\{
		(l, p, q)
		\suchthat*
		\begin{tabular}{l}
			$l \colon x \to d$,~$p \colon x \to d'$ and~$q \colon x \to c''$ \\
			with~$g' p = h'' q$ and~$p = g l$
		\end{tabular}
	\right\}
	\\
	≅{}&
	\left\{
		(l, m)
		\suchthat*
		\begin{tabular}{l}
			$l \colon x \to d$ and~$m \colon x \to c'$ \\
			with~$g' h' m = h'' f' m$ and~$h' m = g l$
		\end{tabular}
	\right\}
	\\
	={}&
	\left\{
		(l, m)
		\suchthat*
		\begin{tabular}{l}
			$l \colon x \to d$ and~$m \colon x \to c'$ \\
			with~$h' m = g l$
		\end{tabular}
	\right\}
\end{align*}
given by
\[
	k
	\mapsto
	(h k, f' f k)
	\mapsto
	(h k, g h k, f' f k)
	\mapsto
	(h k, m)
\]
where~$m$ is the unique morphism from~$x$ to~$c'$ with~$h' m = g h k$ and~$f' m = f' f k$.
The morphism~$f k$ satisfies these defining equations of~$m$, whence~$m = f k$.
The above bijections are thus overall given by
\[
	k \mapsto (h k, f k) \,.
\]
This overall bijection tells us that the left-hand square diagram is a pullback square.
