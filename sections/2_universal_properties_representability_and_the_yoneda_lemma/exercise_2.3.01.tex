\subsection{}

Given a functor~$F \colon \cat{C} \to \Set$ and isomorphism~$F ≅ \cat{C}(c, \ph)$, the universal element corresponding to this isomorphism is the element of~$F c$ that corresponds to the element~$\id_c$ of~$\cat{C}(c, c)$.



\subsubsection{}

Let~$i$ be the unique non-identity morphism in the category~$𝟚$.
The isomorphism~$α \colon \Cat(𝟚, \ph) \to \mor$ is given by
\[
	α_{\cat{C}}(F) = F i \,.
\]
The universal element corresponding to this isomorphism is therefore the element~$α_𝟚(\Id_𝟚) = \Id_𝟚 i = i$ of~$\mor(𝟚)$.



\subsubsection{}

The Sierpiński space~$S$ is given by the two elements~$⊤$ and~$⊥$ and the three open subsets~$∅$,~$\{ ⊤ \}$ and~$S$.
(Its closed subsets are therefore~$∅$,~$\{ ⊥ \}$ and~$S$.)
The isomorphism~$α \colon \Top(\ph, S) \to \Open$ is given by
\[
	α_X(χ) = χ^{-1}(⊤) \,,
\]
i.e., by taking the preimage of the open point.
The universal element corresponding to this isomorphism is therefore the element~$α_S(\id_S) = \id_S^{-1}(⊤) = \{ ⊤ \}$ of~$\Open(S)$, i.e., the open point of~$S$.



\subsubsection{}

We find in the same way as for part~(ii) that the universal element is the element~$\{ ⊥ \}$ of~$\Closed(S)$, i.e., the closed point of~$S$.
