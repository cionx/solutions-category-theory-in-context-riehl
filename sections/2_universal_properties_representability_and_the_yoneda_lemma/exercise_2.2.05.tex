\subsection{}

The set~$Ω$ consists of two elements, whence it admits~$2^2 = 4$ maps into itself.
These maps are as follows:
\begin{enumerate*}

	\item
		The identity map.

	\item
		The transposition that swaps~$⊤$ and~$⊥$.

	\item
		The constant map with value~$⊤$.

	\item
		The constant map with value~$⊥$.

\end{enumerate*}
We denote these maps by~$\id_Ω$,~$σ$,~$c_⊤$ and~$c_⊥$ respectively.

The natural isomorphism between~$P(X)$ and~$\Set(X, Ω)$ is for every set~$X$ given as follows:
\begin{itemize}

	\item
		For every function~$χ \colon X \to Ω$, the corresponding subset of~$X$ is the preimage~$χ^{-1}(⊤)$.

	\item
		For every subset~$A$ of~$X$, the corresponding function is its characteristic function
		\[
			χ_A
			\colon
			X \to Ω \,,
			\quad
			x
			\mapsto
			\begin{cases*}
				⊤ & if~$x ∈ A$, \\
				⊥ & if~$x ∉ A$.
			\end{cases*}
		\]

\end{itemize}
We denote this natural isomorphism from~$\Set(\ph, Ω)$ to~$P$ by~$α$.

\begin{enumerate}

	\item
		The natural endomorphism~$(\id_Ω)_*$ of~$\Set(\ph, Ω)$ is~$\id_{\Set(\ph, Ω)}$.
		The corresponding endomorphism of~$P$ is given by
		\[
			α ⋅ (\id_Ω)_* ⋅ α^{-1}
			=
			α \id_{\Set(\ph, Ω)} α^{-1}
			=
			α α^{-1}
			=
			\id_P \,.
		\]
		In other words, the identity map induces the identity endomorphism.

	\item
		The natural endomorphism of~$P$ induced by~$σ$ is given by~$α σ_* α^{-1}$, and its components are given by
		\begin{align*}
			(α ⋅ σ_* ⋅ α^{-1})_X(A)
			&=
			α_X( (σ_*)_X( α_X^{-1}( A ) ) ) \\
			&=
			α_X( (σ_*)_X( χ_A ) ) \\
			&=
			α_X( σ χ_A ) \\
			&=
			(σ χ_A)^{-1}( ⊤ ) \\
			&=
			χ_A^{-1}( σ^{-1}( ⊤ ) ) \\
			&=
			χ_A^{-1}( ⊥ ) \\
			&=
			X ∖ A \,.
		\end{align*}
		In other words, it is given by taking complements.

	\item
		The natural endomorphism of~$P$ induced by~$c_⊤$ is given by~$α (c_⊤)_* α^{-1}$, and its components are given by
		\begin{align*}
			(α ⋅ (c_⊤)_* ⋅ α^{-1})_X(A)
			&=
			\dotsb \\
			&=
			χ_A^{-1}( c_⊤^{-1}( ⊤ ) ) \\
			&=
			χ_A^{-1}( Ω ) \\
			&=
			X \,.
		\end{align*}
		In other words, it is given by mapping each subset to the entire set.

	\item
		The natural endomorphism of~$P$ induced by~$c_⊥$ is given by~$α (c_⊥)_* α^{-1}$, and its components are given by
		\begin{align*}
			(α ⋅ (c_⊥)_* ⋅ α^{-1})_X(A)
			&=
			\dotsb \\
			&=
			χ_A^{-1}( c_⊥^{-1}( ⊤ ) ) \\
			&=
			χ_A^{-1}( ∅ ) \\
			&=
			∅ \,.
		\end{align*}
		In other words, it is given by mapping each subset to the empty set.

\end{enumerate}
