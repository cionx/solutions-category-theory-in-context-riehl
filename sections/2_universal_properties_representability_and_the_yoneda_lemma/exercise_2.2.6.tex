\subsection{}

\begin{remark}
	The formulation of the question is slightly misleading:
	an \enquote{endomorphism of the category of spaces} is a functor~$\Top \to \Top$, and there are many such functors (e.g., constant functors).
	But the question then asks for something else:
	a non-identity natural endomorphism for an arbitrary topological space.
\end{remark}



\subsubsection*{First solution}

Suppose that we are given for every topological space~$X$ a continuous map
\[
	f_X \colon X \to X
\]
that is natural in~$X$.
More explicitly, this means that for every continuous map~$g \colon X \to Y$ the following square diagram has to commute:
\[
	\begin{tikzcd}
		X
		\arrow{r}[above]{f_X}
		\arrow{d}[left]{g}
		&
		X
		\arrow{d}[right]{g}
		\\
		Y
		\arrow{r}[above]{f_Y}
		&
		Y
	\end{tikzcd}
\]

Let~$X$ be an arbitrary topological space, and for every element~$x$ of~$X$ let~$g_x$ be the map from~$\{ \ast \}$ to~$X$ that picks out the element~$x$, i.e.,
\[
	g_x
	\colon
	\{ \ast \} \to X \,,
	\quad
	\ast \mapsto x \,.
\]
The map~$g_x$ is continuous, whence the diagram
\[
	\begin{tikzcd}
		\{ \ast \}
		\arrow{r}[above]{\id_{\{ \ast \}}}
		\arrow{d}[left]{g_x}
		&
		\{ \ast \}
		\arrow{d}[right]{g_x}
		\\
		X
		\arrow{r}[above]{f_X}
		&
		X
	\end{tikzcd}
\]
commutes.
This commutativity means that
\[
	f_X(x)
	=
	f_X( g_x( \ast ) )
	=
	g_x( \ast )
	=
	x \,.
\]
As these equalities hold for every element~$x$ of~$X$, we find that~$f_X$ is the identity map on~$X$.

The answer to the initial question is therefore \enquote{no}:
there is no non-identity solution.



\subsubsection*{Second solution}

The question at hand is whether the identity functor of~$\Top$ admits a non-trivial endomorphism.
Suppose such an endomorphism~$α$ were to exist.

Let~$U$ be the forgetful functor from~$\Top$ to~$\Set$.
The whiskered natural transformation~$U α$ from~$U \Id_{\Top} = U$ to~$U \Id_{\Top} = U$ is again non-identity because the functor~$U$ is faithful.
More explicitly, there exists some object~$X$ of~$\Top$ for which~$α_X ≠ \id_X$, and then also~$(U α)_X = U α_X ≠ U \id_X = \id_{U X}$.

This shows that every non-identity endomorphism of the identity functor~$\Id_{\Top}$ induces a non-identity endomorphism of the forgetful functor~$U$.

But~$U$ is represented by the singleton space~$\{ \ast \}$.
Therefore, because the Yoneda embedding is full and faithful, endomorphisms of~$U$ are in one-to-one correspondence with endomorphisms of~$\{ \ast \}$.

But~$\{ \ast \}$ admits only the identity endomorphism.
Consequently,~$U$ admits only the identity endomorphism.
Even consequentlier,~$\Id_{\Top}$ admits only the identity endomorphism.
