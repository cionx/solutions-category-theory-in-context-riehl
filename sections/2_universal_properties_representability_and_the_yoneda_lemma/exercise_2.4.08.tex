\subsection{}

For this exercise we denote the morphisms in~$\Elements F$ as quintuples~$(c, x, f, d, y)$ consisting of two objects~$c$ and~$d$ of~$\cat{C}$, elements~$x$ of~$F c$ and~$y$ of~$F d$, and a morphism~$f \colon c \to d$ in~$\cat{C}$ with~$(F f)(x) = y$.
Pictorially,
\[
	F c ∋ x \xto{F f} y ∈ F d \,.
\]
Such a quintuple~$(c, x, f, d, y)$ is a morphism from~$(c, x)$ to~$(d, y)$ in~$\Elements F$.

Let~$φ = (φ_1, φ_2, φ_3, φ_4, φ_5)$ be a morphism in~$\Elements F$.
We make the following observations:
\begin{itemize}

	\item
		That~$φ$ is a lift of~$f$ along~$Π$ is equivalent to the equalities~$φ_1 = c$,~$φ_4 = d$ and~$φ_3 = f$.

	\item
		That the domain of~$φ$ is~$(c, x)$ is equivalent to the two equalities~$φ_1 = c$ and~$φ_2 = x$.

	\item
		The last entry~$φ_5$ is uniquely determined by the previous entries~$φ_2$ and~$φ_3$ as~$φ_5 = (F φ_3)(φ_2)$.

\end{itemize}
Combining all of these observations, we see that~$(c, x, f, d, (F f)(x))$ is the unique lift of~$f$ along~$Π$ whose domain is~$(c, x)$.
