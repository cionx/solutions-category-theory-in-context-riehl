\subsection{}

For the functor~$\cat{C}(A, \ph) × \cat{C}(B, \ph)$ to be representable, we would need an object~$C$ of~$\cat{C}$ such that
\[
	\cat{C}(C, \ph) ≅ \cat{C}(A, \ph) × \cat{C}(B, \ph) \,.
\]
We can also characterize such an isomorphism in terms of its universal element~$(i, j) ∈ \cat{C}(A, C) × \cat{C}(B, C)$:
we would need an object~$C$ of~$\cat{C}$ together with two morphisms
\[
	i \colon A \to C \,,
	\quad
	j \colon B \to C
\]
such that for every object~$X$ of~$\cat{C}$ and every two morphisms~$f \colon A \to X$ and~$g \colon B \to X$ there exists a unique morphism~$h \colon C \to X$ with~$f = h i$ and~$g = h j$.
In terms of a diagram:
\[
	\begin{tikzcd}[column sep = normal]
		A
		\arrow{dr}[above right]{i}
		\arrow[bend right]{ddr}[below left]{f}
		&
		{}
		&
		B
		\arrow{dl}[above left]{j}
		\arrow[bend left]{ddl}[below right]{g}
		\\[-0.5em]
		{}
		&
		C
		\arrow[dashed]{d}[right]{h}
		&
		{}
		\\[1em]
		{}
		&
		X
		&
		{}
	\end{tikzcd}
\]
