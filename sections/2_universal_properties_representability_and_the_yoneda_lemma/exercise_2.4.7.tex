\subsection{}

We will use the following observations:

\begin{lemma}
	\label{lemma for constructing a functor between categories of elements}
	Let~$\cat{C}$ be a category and let~$(\cat{D}, Π)$ and~$(\cat{D}', Π')$ be two categories over~$\cat{C}$, i.e., two objects of~$\CAT / \cat{C}$, for which the functor~$Π'$ is faithful.
	For every object~$d$ of~$\cat{D}$ let~$F d$ be an object of~$\cat{D}'$, and for every morphism~$f \colon d \to d'$ in~$\cat{D}$ let~$F f$ be a morphism in~$\cat{D}'$ from~$F d$ to~$F d'$.
	If the diagram
	\begin{equation}
		\label{auxiliary diagram for functoriality between categories of elements}
		\begin{tikzcd}[column sep = normal]
			\cat{D}
			\arrow{rr}[above]{F}
			\arrow{dr}[below left]{Π}
			&
			{}
			&
			\cat{D}'
			\arrow{dl}[below right]{Π'}
			\\
			{}
			&
			\cat{C}
			&
			{}
		\end{tikzcd}
	\end{equation}
	commutes, then~$F$ is a functor, and more specifically a morphism from~$(\cat{D}, Π)$ to~$(\cat{D}', Π')$ in~$\CAT / \cat{C}$.
\end{lemma}

\begin{proof}
	It remains to show the functoriality of~$F$.
	We hence need to show that
	\[
		F \id_d = \id_{F d}
		\quad\text{and}\quad
		F (g f) = (F g) (F f)
	\]
	for every object~$d$ of~$\cat{D}$, and for every two composable morphisms~$f \colon d \to d'$ and~$g \colon d' \to d''$ in~$\cat{D}$.
	As~$Π'$ is faithful, these two conditions are equivalent to the conditions
	\[
		Π' F \id_d = Π' \id_{F d}
		\quad\text{and}\quad
		Π' F (g f) = Π' ((F g) (F f)) \,.
	\]
	By the functoriality of~$Π'$ we can rewrite (the right-hand sides of) these two equations as follows:
	\[
		Π' F \id_d = \id_{Π' F d}
		\quad\text{and}\quad
		Π' F (g f) = (Π' F g) (Π' F f) \,.
	\]
	By the commutativity of the diagram~\eqref{auxiliary diagram for functoriality between categories of elements} we can simplify these equations to
	\[
		Π \id_d = \id_{Π d}
		\quad\text{and}\quad
		Π (g f) = (Π g) (Π f)) \,.
	\]
	These final equations are satisfied by the functoriality of~$Π$.
\end{proof}

Let~$\cat{C}$ be a category.
For every functor~$F$ from~$\cat{C}$ to~$\Set$ we denote its category of elements by~$\Elements F$, and the forgetful functor from~$\Elements F$ to~$\cat{C}$ by~$Π_F$.
The pair~$(\Elements F, Π_F)$ is an object of the category~$\CAT / \cat{C}$.

Let~$F, G \colon \cat{C} \to \Set$ be two functors and let~$α \colon F \To G$ be a natural transformation.
The natural transformation~$α$ induces a functor
\[
	\Elements α \colon \Elements F \to \Elements G
\]
as follows:
\begin{itemize}

	\item
		Let~$(c, x)$ be an object of~$\Elements F$.
		This means that~$c$ is an object of~$\cat{C}$ and~$x$ is an element of the set~$F c$.
		The component~$α_c$ of the transformation~$α$ is a map from~$F c$ to~$G c$, whence
		\[
			\bigl( \Elements α \bigr) (c, x) ≔ (c, α_c(x))
		\]
		is an object of~$\Elements G$.

	\item
		Let~$f \colon (c, x) \to (c', x')$ be a morphism in~$\Elements F$.
		This means that~$f$ is a morphism in~$\cat{C}$ from~$c$ to~$c'$ with~$(F f)(x) = x'$.
		We have the following commutative square diagram by the naturality of~$α$:
		\[
			\begin{tikzcd}
				F c
				\arrow{r}[above]{F f}
				\arrow{d}[left]{α_c}
				&
				F c'
				\arrow{d}[right]{α_{c'}}
				\\
				G c
				\arrow{r}[above]{G f}
				&
				G c'
			\end{tikzcd}
		\]
		The commutativity of this diagram tells us that
		\[
			(G f)(α_c(x)) = α_{c'}(x) \,,
		\]
		whence~$f$ is a morphism from~$(c, α_c(x))$ to~$(c', α_{c'}(x))$.
		In other words,~$f$ is a morphism from~$(\Elements α) (c, x)$ to~$(\Elements α) (c', x')$.
		We therefore define
		\[
			\bigl( \Elements α \bigr)((c, x) \xto{f} (c', x'))
			≔ ( (c, α_c(x)) \xto{f} (c', α_{c'}(x')) ) \,.
		\]
		For simplicity, we will simply write
		\[
			\bigl( \Elements α \bigr)(f) = f \,,
		\]
		not keeping track of the change in domain and codomain.
		This greatly improves readability, at the minor cost of some rigour.

	\item
		The diagram
		\begin{equation}
			\begin{tikzcd}[column sep = normal]
				\Elements F
				\arrow{rr}[above]{\Elements α}
				\arrow{dr}[below left]{Π_F}
				&
				{}
				&
				\Elements G
				\arrow{dl}[below right]{Π_G}
				\\
				{}
				&
				\cat{C}
				&
				{}
			\end{tikzcd}
		\end{equation}
		commutes, whence it follows from \cref{lemma for constructing a functor between categories of elements} that~$\Elements α$ is a functor from~$\Elements F$ to~$\Elements G$.
\end{itemize}

It remains to show that the induced functor~$\Elements α$ is \emph{itself} functorial in~$α$.
\begin{itemize}

	\item
		We need to show that~$\Elements \id_F = \Id_{\bigl( \Elements F \bigr)}$.
		This holds true because
		\[
			\bigl( \Elements \id_F \bigr) (c, x)
			= (c, (\id_F)_c(x))
			= (c, \id_{F c}(x))
			= (c, x) \,,
		\]
		for every object~$(c, x)$ of~$\Elements F$, as well as
		\[
			\bigl( \Elements \id_F \bigr)(f) = f
		\]
		for every morphism~$f$ in~$\Elements F$.

	\item
		We need to show that~$\Elements (β ⋅ α) = (\Elements β) (\Elements α)$ for every two composable natural transformations~$α \colon F \To G$ and~$β \colon G \To H$.%
		This holds true because
		\begin{align*}
			\bigl( \Elements β \bigr) \bigl( \Elements α \bigr) (c, x)
			&= \bigl( \Elements β \bigr) (c, α_c(x)) \\
			&= (c, β_c(α_c(x))) \\
			&= (c, (β ⋅ α)_c(x)) \\
			&= \bigl( \Elements (β ⋅ α) \bigr) (c, x)
		\end{align*}
		for every object~$(c, x)$ of~$\Elements F$, as well as
		\[
			\bigl( \Elements β \bigr) \bigl( \Elements α \bigr) f
			= \bigl( \Elements β \bigr) f
			= f
			= \bigl( \Elements (β ⋅ α) \bigr) f
		\]
		for every morphism~$f$ in~$\Elements F$.

\end{itemize}

We have overall extended the construction~$\Elements$ to a functor from the functor category~$\Set^{\cat{C}}$ to the slice category~$\CAT / \cat{C}$.
This entails that isomorphic objects of~$\Set^{\cat{C}}$ are mapped to isomorphic objects of~$\CAT / \cat{C}$.
More explicitly, if~$F, G \colon \cat{C} \to \Set$ are two isomorphic functors, then the two categories~$\Elements F$ and~$\Elements G$ are isomorphic over~$\cat{C}$.
