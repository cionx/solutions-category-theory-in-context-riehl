\subsection{}

A dual version of the Yoneda lemma is as follows:
\begin{quote}
	\itshape
	For every contravariant functor~$F \colon \cat{C} \to \Set$, whose domain~$\cat{C}$ is locally small, and every object~$c$ of~$\cat{C}$, the map
	\[
		Φ \colon \Hom( \cat{C}(\ph, c), F )
		\to
		F c \,,
		\quad
		α
		\mapsto
		α_c(\id_c)
	\]
	is bijective and natural in both~$c$ and~$F$.
\end{quote}



\subsubsection*{Injectivity of~$Φ$}

To see that the map~$Φ$ is injective let~$α \colon \cat{C}(\ph, c) \To F$ be a natural isomorphism, let~$x$ be an arbitrary object of~$\cat{C}$ and let~$f$ be an arbitrary element of~$\cat{C}(\ph, c)(x)$.
As~$\cat{C}(\ph, c)(x) = \cat{C}(x, c)$, this means that~$f$ is a morphism from~$x$ to~$c$ in~$\cat{C}$.
It follows from the commutativity of the square diagram
\[
	\begin{tikzcd}
		\cat{C}(c, c)
		\arrow{r}[above]{f^*}
		\arrow{d}[left]{α_c}
		&
		\cat{C}(x, c)
		\arrow{d}[right]{α_x}
		\\
		F c
		\arrow{r}[above]{F f}
		&
		F x
	\end{tikzcd}
\]
that
\[
	α_x( f )
	=
	α_x( \id_c f )
	=
	α_x( f^*( \id_c ) )
	=
	(F f)( α_c( \id_c ) )
	=
	(F f)( Φ(α) ) \,.
\]
This shows that the entire natural transformation~$α$ is uniquely determined by the single element~$Φ(α)$.



\subsubsection*{Surjectivity of~$Φ$}

Let~$u$ be an arbitrary element of the set~$F c$.
We consider for every object~$x$ of~$\cat{C}$ the map
\[
	α_x
	\colon
	\cat{C}(x, c) \to F x \,,
	\quad
	g \mapsto (F g)(u) \,.
\]
We claim that the family~$α ≔ (α_x)_{x ∈ \cat{C}}$ is a natural transformation from~$\cat{C}(\ph, c)$ to~$F$ with~$Φ(α) = u$.

To see that~$α$ is a natural transformation from~$\cat{C}(\ph, c)$ to~$F$, we consider an arbitrary morphism~$f \colon x \to y$ in~$\cat{C}$ and the resulting commutative diagram:
\[
	\begin{tikzcd}
		\cat{C}(y, c)
		\arrow{r}[above]{f^*}
		\arrow{d}[left]{α_y}
		&
		\cat{C}(x, c)
		\arrow{d}[right]{α_x}
		\\
		F y
		\arrow{r}[above]{F f}
		&
		F x
	\end{tikzcd}
\]
This diagram commutes because
\[
	α_x( f^*( g ) )
	=
	α_x( g f )
	=
	F(g f)(u)
	=
	(F f)( (F g)( u ) )
	=
	(F f)( α_y( g ) )
\]
for every element~$g$ of~$\cat{C}(y, c)$, i.e., every morphism~$g \colon c \to y$ in~$\cat{C}$.
This shows the naturality of~$α$.

We also have~$Φ(α) = α_c(\id_c) = (F \id_c)(u) = \id_{F(c)}(u) = u$, as desired.



\subsubsection*{Naturality in~$c$}

We relabel the bijection~$Φ$ as~$Φ_c$, and want to show that~$Φ_c$ is natural in~$c$.

Let~$f \colon c \to d$ be a morphism in~$\cat{C}$.
We need to show that the square diagram
\begin{equation}
	\label{first naturality of contravariant yoneda}
	\begin{tikzcd}
		\Hom(\cat{C}(\ph, d), F)
		\arrow{r}[above]{(f_*)^*}
		\arrow{d}[left]{Φ_d}
		&
		\Hom(\cat{C}(\ph, c), F)
		\arrow{d}[right]{Φ_c}
		\\
		F d
		\arrow{r}[above]{F f}
		&
		F c
	\end{tikzcd}
\end{equation}
commutes.
To this end let~$α$ be an element in the top-left corner of this diagram, i.e., a natural transformation from~$\cat{C}(\ph, d)$ to~$F$.
One path from the top-left corner to the bottom-right corner in the diagram~\eqref{first naturality of contravariant yoneda} equals
\begin{align*}
	Φ_c( (f_*)^*( α ) )
	&=
	Φ_c( α ⋅ f_* ) \\
	&=
	(α ⋅ f_*)_c( \id_c ) \\
	&=
	(α_c ∘ (f_*)_c)(\id_c) \\
	&=
	α_c( (f_*)_c( \id_c ) ) \\
	&=
	α_c( f \id_c ) \\
	&=
	α_c( f ) \,.
\end{align*}
For the other path from the top-left corner to the bottom-right corner we observe that the diagram
\[
	\begin{tikzcd}
		\cat{C}(d, d)
		\arrow{r}[above]{f^*}
		\arrow{d}[left]{α_d}
		&
		\cat{C}(c, d)
		\arrow{d}[right]{α_c}
		\\
		F d
		\arrow{r}[above]{F f}
		&
		F c
	\end{tikzcd}
\]
commutes by the naturality of~$α$, whence
\[
	(F f)( Φ_d( α ) )
	=
	(F f)( α_d(\id_d) )
	=
	α_c( f^*( \id_d ) )
	=
	α_c( \id_d f )
	=
	α_c( f ) \,.
\]
This shows the commutativity of the diagram~\eqref{first naturality of contravariant yoneda}.



\subsubsection*{Naturality in~$F$}

We relabel the bijection~$Φ$ as~$Φ_F$, and want to show that~$Φ_F$ is natural in~$F$.

Let~$F$ and~$G$ be two contravariant functors from~$\cat{C}$ to~$\Set$, and let~$β \colon F \To G$ be a natural transformation.
We need to show that the square diagram
\[
	\begin{tikzcd}
		\Hom(\cat{C}(\ph, c), F)
		\arrow{r}[above]{β_*}
		\arrow{d}[left]{Φ_F}
		&
		\Hom(\cat{C}(\ph, c), G)
		\arrow{d}[right]{Φ_G}
		\\
		F c
		\arrow{r}[above]{β_c}
		&
		G c
	\end{tikzcd}
\]
commutes.
To this end let~$α$ be an element of the top-left corner of this diagram, i.e., let~$α$ be a natural transformation from~$\cat{C}(\ph, c)$ to~$F$.
The sequence of equalities
\[
	Φ_G( β_*( α ) )
	=
	Φ_G( β ⋅ α )
	=
	(β ⋅ α)_c( \id_c )
	=
	(β_c α_c)( \id_c )
	=
	β_c( α_c( \id_c ) )
	=
	β_c( Φ_F( α ) )
\]
tells us that the above diagram indeed commutes.
