\subsection{}

The isomorphism
\[
	α
	\colon
	\Set(\ph, B^A)
	\cong
	\Set(\ph × A, B)
\]
is explicitly given by
\begin{equation}
	\label{explicit description of currying}
	α_X(φ)(x, a) = φ(x)(a)
\end{equation}
for every set~$X$, every function~$φ \colon X \to B^A$ and all~$(x, a) ∈ X × A$.

The universal element~$\ev$ corresponding to the isomorphism~$α$ is given by~$α_{B^A}(\id_{B^A})$.
It is thus an element of~$\Set(B^A × A, B)$, i.e., a map from~$B^A × A$ to~$B$.
It follows from the explicit formula~\eqref{explicit description of currying} that this map is given by
\[
	\ev(f, a)
	=
	α_{B^A}( \id_{B^A} )(f, a)
	=
	\id_{B^A}(f)(a)
	=
	f(a)
\]
for all~$(f, a) ∈ B^A × A$.
The map~$\ev$ is thus given by evaluation of the first item at the second item.

The Yoneda lemma tells us for the functor~$F ≔ \Set(\ph × A, B)$ that the entire natural transformation~$α \colon \Set(\ph, B^A) \To F$ is uniquely determined by the element~$\ev$ of~$F(B^A)$.
More explicitly,~$α_X(φ) = (F φ)(\ev)$ for every set~$X$, and therefore
\[
	α_X(φ) = (F φ)(\ev) = {\ev} ⋅ (φ × \id_A) \,.
\]
That~$α$ is a natural isomorphism means that~$α_X$ is bijective for every set~$X$.
In other words:
\begin{quote}
	For every set~$X$ and every element~$f$ of~$F X$ there exists a unique element~$φ$ of~$\Set(X, B^A)$ with~$α_X(φ) = f$.
\end{quote}
We can expand this condition as follows:
\begin{quote}
	For every set~$X$ and every map~$f \colon X × A \to B$, there exists a unique map~$φ \colon X \to B^A$ such that~$f(c, a) = \ev(φ(c), a)$ for all~$(c, a) ∈ X × A$, i.e., such that the following diagram commutes:
	\[
		\begin{tikzcd}
			X × A
			\arrow{r}[above]{f}
			\arrow{d}[left]{φ × \id_A}
			&
			B
			\\
			B^A × A
			\arrow{ur}[below right]{\ev}
			&
			{}
		\end{tikzcd}
	\]
\end{quote}
