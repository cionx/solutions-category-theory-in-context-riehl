\subsection{}

The path functor is represented by the unit interval.
It follows from the Yoneda embedding that each automorphism of the path functor is induced by an automorphism of the unit interval.

More explicitly, suppose that we have for every topological space~$X$ a reparametrization procedure
\[
	α_X \colon \Path(X) \to \Path(X) \,,
\]
and that this procedure is natural in~$X$.
Then there exists a homeomorphism~$r$ of the unit interval~$I$ such that
\[
	α_X(γ) = γ r
\]
for every path~$γ$ in~$X$.

We can further characterize homeomorphisms of the unit interval.

\begin{proposition}
	Every homeomorphism of the unit interval is strictly monotone, i.e., strictly increasing or strictly decreasing.
\end{proposition}

\begin{proof}
	Let~$f$ be a homeomorphism of~$I$.
	The two endpoints~$0$ and~$1$ of~$I$ are the only two points in~$I$ whose removal does \emph{not} split the interval into two path components.
	Consequently,~$f$ needs to permute these two endpoints.
	So either~$f(0) = 0$ and~$f(1) = 1$, or~$f(0) = 1$ and~$f(1) = 0$.
	We can switch between the two cases by post-composing~$f$ with the flip map~$σ \colon x \mapsto 1 - x$.
	Also,~$f$ is strictly decreasing if and only if~$σ f$ is strictly increasing.
	It therefore suffices to consider the first case.

	We observe that if~$x$ and~$y$ are two points in~$I$ with~$x < y$ and~$f(x) < f(y)$, then we have~$f(x) < f(z) < f(y)$ for every~$z$ with~$x < z < y$.
	Indeed, suppose otherwise.
	Then either~$f(z) ≤ f(x) < f(y)$ or~$f(x) < f(y) ≤ f(z)$.
	\begin{itemize*}

		\item
			If~$f(z) ≤ f(x) < f(y)$, then it follows from the intermediate value theorem that
			\[
				f(x) ∈ [f(z), f(y)] ⊆ f([z, y])
			\]
			even though~$x$ is not contained in the interval~$[z, y]$.
			This contradicts the injectivity of~$f$.

		\item
			If~$f(x) < f(y) ≤ f(z)$, then it follows from the intermediate value theorem that
			\[
				f(y) ∈ [f(x), f(z)] ⊆ f([x, z])
			\]
			even though~$y$ is not contained in the interval~$[x, z]$.
			This contradicts the injectivity of~$f$.

	\end{itemize*}
	This shows that indeed~$f(x) < f(z) < f(y)$.

	Let now~$x, y ∈ I$ with~$x < y$.
	\begin{itemize*}
	
		\item
			If~$x = 0$, then~$0 < y$, therefore~$f(y) ≠ f(0) = 0$, thus~$f(y) > 0 = f(x)$.

		\item
			If~$x ≠ 0$, then~$0 < x < y$ and thus~$f(0) < f(x) < f(y)$, which entails that~$f(x) < f(y)$.

	\end{itemize*}
	We find in every case that~$f(x) < f(y)$.
\end{proof}

\begin{lemma}
	Every strictly monotone, surjective map from~$I$ into itself is a homeomorphism.
\end{lemma}

\begin{proof}
	Let~$f$ be such a map.
	We may post-compose the map~$f$ with the flip map~$x \mapsto 1 - x$ to assume that~$f$ is strictly increasing.

	The strict monotonicity of~$f$ ensures that~$f$ is injective.
	Together with the surjectivity of~$f$, this tells us that~$f$ is bijective.
	The inverse of~$f$ is again strictly increasing.
	It hence suffices to show that under the given conditions,~$f$ is continuous.
	(As swapping the roles of~$f$ and~$f^{-1}$ will then also show that~$f^{-1}$ is continuous.)

	We have~$f(0) ≤ x$ for every~$x ∈ I$, therefore~$f(0) ≤ f(x)$ for every~$x ∈ I$, and thus~$f(0) = 0$ because~$f$ is surjective.
	We find in the same way that also~$f(1) = 1$.

	It follows that the extended map
	\[
		g
		\colon
		ℝ \to ℝ \,,
		\quad
		x
		\mapsto
		\begin{cases*}
			x     & if~$x ≤ 0$, \\
			f(x)  & if~$x ∈ I$, \\
			x     & if~$x ≥ 1$,
		\end{cases*}
	\]
	is continuous if and only if the original map~$f$ is continuous.
	The map~$g$ is again strictly increasing and bijective.
	We show in the following that~$g$ is continuous.

	Let~$x ∈ ℝ$ and let~$ε > 0$.
	There exists for~$y ≔ g(x)$ two points~$y_1, y_2 ∈ ℝ$ with
	\[
		y - ε < y_1 < y < y_2 < y + ε \,.
	\]
	It follows for the points~$x_1 ≔ f^{-1}(y_1)$ and~$x_2 ≔ f^{-1}(y_2)$ that~$x_1 < x < x_2$ because the inverse map~$f^{-1}$ is again strictly increasing.
	There hence exists some~$δ > 0$ with~$(x - δ, x + δ) ⊆ [x_1, x_2]$.
	We have~$f([x_1, x_2]) ⊆ [f(x_1), f(x_2)]$ because~$f$ is increasing, and therefore
	\[
		f( (x + δ, x - δ) )
		⊆
		f( [x_1, x_2] )
		⊆
		[f(x_1), f(x_2)]
		=
		[y_1, y_2]
		⊆
		(y - ε, y + ε) \,.
	\]
	As~$ε > 0$ was arbitrary, this shows that~$f$ is continuous.
\end{proof}

\begin{corollary}
	The homeomorphisms of the unit interval~$I$ are precisely those maps from~$I$ into itself that are strictly monotone and surjective.

	Equivalently, the homeomorphisms are precisely those maps~$f \colon I \to I$ that are continuous, and either strictly increasing with~$f(0) = 0$ and~$f(1) = 1$ or strictly decreasing with~$f(0) = 1$ and~$f(1) = 0$.
	\qed
\end{corollary}
