\subsection{}

The functor category~$\cat{C} ≔ \Set^{(\bbgreek{ω}^{\op})}$ can be regarded as the category of contravariant functors from~$\bbgreek{ω}$ to~$\Set$.
This category can (up to isomorphism) more explicitly be described as follows:
\begin{itemize}

	\item
		The objects of~$\cat{C}$ are diagrams of the form
		\[
			\begin{tikzcd}
				A_0
				&
				A_1
				\arrow{l}[above]{a_0}
				&
				A_2
				\arrow{l}[above]{a_1}
				&
				A_3
				\arrow{l}[above]{a_2}
				&
				\arrow{l}
				\dotsb
			\end{tikzcd}
		\]
		consisting of sets and maps between them.

	\item
		A morphism from an object~$((A_n)_n, (a_n)_n)$ to an object~$((B_n)_n, (b_n)_n)$ is a family~$(f_n)_n$ of maps~$f_n \colon A_n \to B_n$ such that the following diagram commutes:
		\[
			\begin{tikzcd}
				A_0
				\arrow{d}[left]{f_0}
				&
				A_1
				\arrow{l}[above]{a_0}
				\arrow{d}[left]{f_1}
				&
				A_2
				\arrow{l}[above]{a_1}
				\arrow{d}[left]{f_2}
				&
				A_3
				\arrow{l}[above]{a_2}
				\arrow{d}[left]{f_3}
				&
				\arrow{l}
				\dotsb
				\\
				B_0
				&
				B_1
				\arrow{l}[above]{b_0}
				&
				B_2
				\arrow{l}[above]{b_1}
				&
				B_3
				\arrow{l}[above]{b_2}
				&
				\arrow{l}
				\dotsb
			\end{tikzcd}
		\]

\end{itemize}

For every object~$n$ of~$\bbgreek{ω}$, the object~$よ(n)$ consists of~$n + 1$ many singleton sets, followed by empty sets:%
\footnote{
	We denote the Yoneda embedding by~$よ$ instead of~$y$.
}
\[
		\{ \ast \}
		\from
		\{ \ast \}
		\from
		\dotsb
		\from
		\{ \ast \}
		\from
		∅
		\from
		∅
		\from
		\dotsb
\]
We know that in~$\bbgreek{ω}$ there exists a unique morphism from~$n$ to~$m$ if~$n ≤ m$, and no morphism if~$n > m$.
It suffices to show that the same holds true for the induced diagrams~$よ(n)$ and~$よ(m)$.
\begin{itemize}

	\item
		Suppose that~$n ≤ m$.
		Then there exists precisely one sequence of maps~$(f_n)_n$ that makes the following diagram commute:
		\[
			\begin{tikzcd}[column sep = small]
				よ(n)
				\colon
				&
				\{ \ast \}
				\arrow{d}[left]{f_0}
				&
				⋯
				\arrow{l}
				\arrow[phantom]{d}[description]{⋰}
				&
				\{ \ast \}
				\arrow{l}
				\arrow{d}[left]{f_n}
				&
				∅
				\arrow{l}
				\arrow{d}[left]{f_{n + 1}}
				&
				⋯
				\arrow{l}
				\arrow[phantom]{d}[description]{⋰}
				&
				∅
				\arrow{l}
				\arrow{d}[left]{f_m}
				&
				∅
				\arrow{l}
				\arrow{d}[left]{f_{m + 1}}
				&
				⋯
				\arrow{l}
				\\
				よ(m)
				\colon
				&
				\{ \ast \}
				&
				⋯
				\arrow{l}
				&
				\{ \ast \}
				\arrow{l}
				&
				\{ \ast \}
				\arrow{l}
				&
				⋯
				\arrow{l}
				&
				\{ \ast \}
				\arrow{l}
				&
				∅
				\arrow{l}
				&
				⋯
				\arrow{l}
			\end{tikzcd}
		\]
		Indeed, the maps~$f_0, \dotsc, f_n$ are necessarily the identity maps on the singleton set, and~$f_k$ for~$k > n$ is necessarily the empty map.

	\item
		Suppose now that~$n > m$.
		A morphism~$f \colon よ(n) \to よ(m)$ contains a map~$f_{m + 1} \colon \{ \ast \} \to ∅$, but such a map does not exist.
		Consequently, no morphism from~$よ(n)$ to~$よ(m)$ exists.

\end{itemize}
