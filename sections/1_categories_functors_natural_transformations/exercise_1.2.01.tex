\subsection{}

The exercise tasks us with proving that two categories are isomorphic, even though the book has yet to introduce the notion of an isomorphism of categories.
We will construct in the following a contravariant functor from~$\cat{C} / c$ to~$c / \cat{C}$ that is bijective on objects and on morphisms.

Let~$c$ be an object in a category~$\cat{C}$.

The objects of the slice category~$\cat{C} / c$ are the pairs~$(x, f)$ consisting of another object~$x$ of~$\cat{C}$ and a morphism~$f \colon c \to x$ in~$\cat{C}$.
Similarly, the objects of the slice category~$c / \cat{C}^{\op}$ are the pairs~$(x, f')$ consisting of another object~$x$ of~$\cat{C}^{\op}$ and a morphism~$f' \colon x \to c$ in~$\cat{C}^{\op}$.
The two categories~$\cat{C}$ and~$\cat{C}^{\op}$ have the same objects, and we know that morphisms in~$\cat{C}^{\op}$ correspond bijectively to morphisms in~$\cat{C}$ via the mapping~$(f \colon x \to y) \mapsto (f^{\op} \colon y \to x)$.
It follows that this bijection restricts to a bijection between the objects of~$\cat{C} / c$ on the one hand and the objects of~$c / \cat{C}^{\op}$ on the other hand.
More explicitly, this bijection is given by
\[
	(x, {\textstyle f \colon x \to c})
	\mapsto
	(x, {\textstyle f^{\op} \colon c \to x}) \,.
\]

Let~$(x, f)$ and~$(y, g)$ be two objects of~$\cat{C}/ c$.
A morphism from~$(x, f)$ to~$(y, g)$ in~$\cat{C} / c$ is a morphism~$φ$ from~$x$ to~$y$ in~$\cat{C}$ for which the following diagram commutes:
\[
	\begin{tikzcd}[column sep = normal]
		x
		\arrow{rr}[above]{φ}
		\arrow{dr}[below left]{f}
		&
		{}
		&
		y
		\arrow{dl}[below right]{g}
		\\
		{}
		&
		c
		&
		{}
	\end{tikzcd}
\]
Similarly, a morphism from~$(y, g^{\op})$ to~$(x, f^{\op})$ in~$c / \cat{C}^{\op}$ is a morphism~$φ'$ from~$y$ to~$x$ in~$\cat{C}^{\op}$ for which the following diagram commutes:
\[
	\begin{tikzcd}[column sep = normal]
		{}
		&
		c
		\arrow{dl}[above left]{g^{\op}}
		\arrow{dr}[above right]{f^{\op}}
		&
		{}
		\\
		y
		\arrow{rr}[above]{φ'}
		&
		{}
		&
		x
	\end{tikzcd}
\]
We have thus for every morphism~$φ \colon x \to y$ in~$\cat{C}$ the sequence of equivalences
\begin{align*}
	{}&
	\text{$φ$ is a morphism from~$(x, f)$ to~$(y, g)$ in~$\cat{C} / c$}
	\\
	\iff{}&
	g φ = f
	\\
	\iff{}&
	(g φ)^{\op} = f^{\op}
	\\
	\iff{}&
	φ^{\op} g^{\op} = f^{\op}
	\\
	\iff{}&
	\text{$φ^{\op}$ is a morphism from~$(y, g^{\op})$ to~$(x, f^{\op})$ in~$c / \cat{C}^{\op}$} \,.
\end{align*}
These equivalences tell us that the bijection between morphisms of~$\cat{C}$ and morphisms of~$\cat{C}^{\op}$ given by~$φ \mapsto φ^{\op}$ restricts for every two objects~$(x, f)$ and~$(y, g)$ of~$\cat{C} / c$ to a bijection between morphisms from~$(x, f)$ to~$(y, g)$ in~$\cat{C} / c$ on the one hand and morphisms from~$(y, g^{\op})$ to~$(x, f^{\op})$ in~$c / \cat{C}^{\op}$ on the other hand:
\[
	\begin{tikzcd}[column sep = normal]
		x
		\arrow{rr}[above]{φ}
		\arrow{dr}[below left]{f}
		&
		{}
		&
		y
		\arrow{dl}[below right]{g}
		\\
		{}
		&
		c
		&
		{}
	\end{tikzcd}
	\qquad\onetoone\qquad
\begin{tikzcd}[column sep = normal]
		x
		&
		{}
		&
		\arrow{ll}[above]{φ^{\op}}
		y
		\\
		{}
		&
		c
		\arrow{ul}[below left]{f^{\op}}
		\arrow{ur}[below right]{g^{\op}}
		&
		{}
	\end{tikzcd}
\]

We have thus constructed a bijection between the objects of the two categories~$\cat{C} / c$ and~$c / \cat{C}^{\op}$, given by~$(x, f) \mapsto (x, f^{\op})$, as well as for every two objects~$(x, f)$ and~$(y, g)$ of~$\cat{C} / c$ a bijection between the morphisms from~$(x, f)$ to~$(y, g)$ in~$\cat{C} / c$ and the morphisms from~$(y, g^{\op})$ to~$(x, f^{\op})$ in~$c / \cat{C}^{\op}$, given by~$φ \mapsto φ^{\op}$.
We denote these mappings by~$D$ (for \enquote{duality}).

It remains to check the functoriality of~$D$.
We have for every object~$(x, f)$ of~$\cat{C} / c$ the sequence of equalities
\[
	D(\id_{(x, f)})
	= \id_{(x, f)}^{\op}
	= \id_{x, \cat{C}}^{\op}
	= \id_{x, \cat{C}^{\op}}
	= \id_{(x, f^{\op})}
	= \id_{D((x, f))} \,,
\]
which shows that~$D$ preserves identities.
We also have for any two composable morphisms~$φ \colon (x, f) \to (y, g)$ and~$ψ \colon (y, g) \to (z, h)$ in~$\cat{C} / c$ the sequence of equalities
\[
	D(ψ φ)
	=
	(ψ φ)^{\op}
	=
	φ^{\op} ψ^{\op}
	=
	D(φ) D(ψ) \,,
\]
which shows that~$D$ contravariantly preserves composition.

Regarding the second part of this exercise:
By using the contravariant isomorphism~$D$, one could actually define~$\cat{C} / c$ as~$(c / \cat{C}^{\op})^{\op}$.
This would then allow us to deduce part~(ii) of Exercise~1.1.iii from the previous part~(i).
