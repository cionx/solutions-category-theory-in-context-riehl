\subsection{}

We first prove an auxiliary result.

\begin{proposition}
	\label{natural transformations via diagonal morphisms}
	Let~$\cat{C}$ and~$\cat{D}$ be categories and let~$F, G \colon \cat{C} \to \cat{D}$ be functors.
	\begin{enumerate}

		\item
			Let~$α \colon F \To G$ be a natural transformation.
			For every morphism~$f \colon x \to y$ in~$\cat{C}$ let~$η_f \colon F(x) \to G(y)$ be the diagonal morphism in the following commutative square diagram:
			\begin{equation}
				\label{construction of diagonal transformation}
				\begin{tikzcd}[row sep = huge]
					F(x)
					\arrow{r}[above]{F(f)}
					\arrow{d}[left]{α_x}
					\arrow[dashed]{dr}[above right]{η_f}
					&
					F(y)
					\arrow{d}[right]{α_y}
					\\
					G(x)
					\arrow{r}[below]{G(f)}
					&
					G(y)
				\end{tikzcd}
			\end{equation}
			In other words,~$η_f = α_y ∘ F(f)$ and also~$η_f = G(f) ∘ α_x$.

			Then the following triangular diagrams commute for all composable morphisms~$f \colon x \to y$ and~$g \colon y \to z$ in~$\cat{C}$:
			\[
				\begin{tikzcd}[column sep = normal]
					{}
					&
					F(x)
					\arrow{dl}[above left]{η_f}
					\arrow{dr}[above right]{η_{g f}}
					&
					{}
					\\
					G(y)
					\arrow{rr}[above]{G(g)}
					&
					{}
					&
					G(z)
				\end{tikzcd}
				\qquad
				\begin{tikzcd}[column sep = normal]
					F(x)
					\arrow{rr}[above]{F(f)}
					\arrow{dr}[below left]{η_{gf}}
					&
					{}
					&
					F(y)
					\arrow{dl}[below right]{η_g}
					\\
					{}
					&
					G(z)
					&
					{}
				\end{tikzcd}
			\]
			In other words, we have
			\begin{equation}
				\label{algebraic condition for diagonal natural transformation}
				η_{gf} = G(g) ∘ η_f
				\quad\text{and}\quad
				η_{gf} = η_g ∘ F(f) \,.
			\end{equation}

		\item
			Suppose conversely that~$(η_f)_f$ is a family of morphisms~$η_f \colon F(x) \to G(y)$, where~$f \colon x \to y$ ranges through the morphisms in~$\cat{C}$, such that the conditions~\eqref{algebraic condition for diagonal natural transformation} hold for all morphisms~$f \colon x \to y$ and~$g \colon y \to z$ in~$\cat{C}$.

			Then there exists a unique natural transformation~$α \colon F \To G$ such that the diagram~\eqref{construction of diagonal transformation} commutes for every morphism~$f \colon x \to y$ in~$\cat{C}$.

		\item
			The above two constructions are mutually inverse, and thus result in a bijection between
			\begin{itemize}

				\item
					natural transformations~$α \colon F \To G$ and

				\item
					families~$(η_f)_f$ of morphisms~$η_f \colon F(x) \to G(y)$, where~$f \colon x \to y$ ranges through the morphisms in~$\cat{C}$, such that~$η_{gf} = G(g) ∘ η_f$ and~$η_{gf} = η_g ∘ F(f)$ for all morphisms~$f \colon x \to y$ and~$g \colon y \to z$ in~$\cat{C}$.

			\end{itemize}

	\end{enumerate}
\end{proposition}

\begin{proof}
	\leavevmode
	\begin{enumerate}

		\item
			We have for all morphisms~$f \colon x \to y$ and~$g \colon y \to z$ in~$\cat{C}$ the following two commutative diagrams:
			\[
				\begin{tikzcd}
					F(x)
					\arrow[bend left = 40]{rr}[above]{F(g ∘ f)}
					\arrow{r}[above]{F(f)}
					\arrow{d}[left]{α_x}
					\arrow[dashed]{dr}[above right]{η_f}
					&
					F(y)
					\arrow{r}[above]{F(g)}
					\arrow{d}[right]{α_y}
					\arrow[dashed]{dr}[above right]{η_g}
					&
					F(z)
					\arrow{d}[right]{α_z}
					\\
					G(x)
					\arrow{r}[below]{G(f)}
					\arrow[bend right = 40]{rr}[below]{G(g ∘ f)}
					&
					G(y)
					\arrow{r}[below]{G(g)}
					&
					G(z)
				\end{tikzcd}
				\qquad
				\begin{tikzcd}
					F(x)
					\arrow{r}[above]{F(g ∘ f)}
					\arrow{d}[left]{α_x}
					\arrow[dashed]{dr}[above right]{η_{gf}}
					&
					F(z)
					\arrow{d}[right]{α_z}
					\\
					G(x)
					\arrow{r}[below]{G(g ∘ f)}
					&
					G(z)
				\end{tikzcd}
			\]
			The outer square part of these diagrams is the same.
			The overall diagonal morphism from~$F(x)$ to~$G(z)$ is therefore the same in both diagrams.
			This entails that~$η_{gf} = η_g ∘ F(f)$ as well as~$η_{gf} = G(g) ∘ η_f$.

		\item
			If then such a natural transformation~$α$ were to exist, then we would have for every object~$x$ of~$\cat{C}$ the following commutative square diagram:
			\[
				\begin{tikzcd}[row sep = huge]
					F(x)
					\arrow{r}[above]{F(\id_x)}
					\arrow{d}[left]{α_x}
					\arrow[dashed]{dr}[above right]{η_{\id_x}}
					&
					F(x)
					\arrow{d}[right]{α_x}
					\\
					G(x)
					\arrow{r}[below]{G(\id_x)}
					&
					G(x)
				\end{tikzcd}
			\]
			It would then follow from the identity~$F(\id_x) = \id_{F(x)}$ (or~$G(\id_x) = \id_{G(x)}$) that~$α_x = η_{\id_x}$.
			This shows the uniqueness of~$α$.

			To prove the existence of~$α$ we now set
			\[
				α_x ≔ η_{\id_x}
			\]
			for every object~$x$ of~$\cat{C}$.
			We need to show that the family~$α ≔ (α_x)_x$ is a natural transformation from~$F$ to~$G$, and that~$η_f = α_y ∘ F(f)$ and~$η_f = G(f) ∘ α_x$ for every morphism~$f \colon x \to y$ in~$\cat{C}$.
			These equalities hold because
			\[
				η_f = η_{\id_y f} = η_{\id_y} ∘ F(f) = α_y ∘ F(f)
			\]
			and similarly
			\[
				η_f = η_{f \id_x} = G(f) ∘ η_{\id_x} = G(f) ∘ α_x \,.
			\]
			The resulting concatenation of equalities~$α_y ∘ F(f) = η_f = G(f) ∘ α_x$ also shows the naturality of~$α$.

		\item
			Let~$α \colon F \To G$ be a natural transformation.
			Let~$(η_f)_f$ be the resulting family of morphisms~$η_f \colon F(x) \to G(y)$, where~$f \colon x \to y$ ranges through the morphisms in~$\cat{C}$, given by~$η_f = α_y ∘ F(f)$ (and also~$η_f = G(f) ∘ α_x$).
			Let~$α'$ be the resulting natural transformation from~$F$ to~$G$ given by~$α'_x = η_{\id_x}$ for every object~$x$ of~$\cat{C}$.
			We then have for every object~$x$ of~$\cat{C}$ the sequence of equalities
			\[
				α'_x = η_{\id_x} = α_x ∘ F(\id_x) = α_x ∘ \id_{F(x)} = α_x \,,
			\]
			which shows that~$α' = α$.

			Let now conversely~$(η_f)_f$ be a family of morphisms~$η_f \colon F(x) \to G(y)$, where~$f \colon x \to y$ ranges through the morphisms in~$\cat{C}$, satisfying the two conditions~$η_{gf} = G(g) ∘ η_f$ and~$η_{gf} = η_g ∘ F(f)$ for all morphisms~$f \colon x \to y$ and~$g \colon y \to z$ in~$\cat{C}$.
			Let~$α$ be the resulting natural transformation from~$F$ to~$G$ given by~$α_x = η_{\id_x}$ for every object~$x$ of~$\cat{C}$.
			Let~$(η'_f)_f$ be the resulting family of morphisms~$η'_f \colon F(x) \to G(y)$, where once again~$f \colon x \to y$ ranges through the morphisms in~$\cat{C}$, with~$η'_f$ given by~$η'_f = α_y ∘ F(f)$ (and also equivalently~$η'_f = G(f) ∘ α_x$).
			Then
			\[
				η'_f = α_y ∘ F(f) = η_{\id_y} ∘ F(f) = η_{\id_y f} = η_f
			\]
			for every morphism~$f \colon x \to y$ in~$\cat{C}$.
		\qedhere

	\end{enumerate}
\end{proof}

We now return to the exercise at hand.
We note that a functor
\[
	H \colon \cat{C} × 𝟚 \to \cat{D}
\]
consists of the following data:
\begin{enumerate*}[label = D\arabic*., ref = D\arabic*]

	\item
		\label{data of H(x, 0)}
		For every object~$x$ of~$\cat{C}$ an object~$H(x, 0)$ of~$\cat{D}$.

	\item
		\label{data of H(x, 1)}
		For every object~$x$ of~$\cat{C}$ an object~$H(x, 1)$ of~$\cat{D}$.

	\item
		\label{data of H(f, id0)}
		For every morphism~$f \colon x \to y$ in~$\cat{C}$ a morphism~$H(f, \id_0)$ from~$H(x, 0)$ to~$H(y, 0)$ in~$\cat{D}$.

	\item
		\label{data of H(f, id1)}
		For every morphism~$f \colon x \to y$ in~$\cat{C}$ a morphism~$H(f, \id_1)$ from~$H(x, 1)$ to~$H(y, 1)$ in~$\cat{D}$.

	\item
		\label{data of H(f, j)}
		For every morphism~$f \colon x \to y$ in~$\cat{C}$ a morphism~$H(f, j)$ from~$H(x, 0)$ to~$H(y, 1)$ in~$\cat{D}$, where~$j \colon 0 \to 1$ is the unique non-identity morphism in~$𝟚$.

\end{enumerate*}
These data are subject to the following conditions:
\begin{enumerate*}[label = C\arabic*., ref = C\arabic*]

	\item
		For every object~$x$ of~$\cat{C}$ the two equalities
		\begin{enumerate}[label = C1.\alph*., ref = C1.\alph*]
		% I don’t know how to automate the number 1.

			\item
				\label{identities in the first coordinate}
				$H(\id_x, \id_0) = \id_{H(x, 0)}$ and

			\item
				\label{identities in the second coordinate}
				$H(\id_x, \id_1) = \id_{H(x, 1)}$.

		\end{enumerate}

	\item
		For all morphisms~$f \colon x \to y$ and~$g \colon y \to z$ in~$\cat{C}$ the four equalities
		\begin{enumerate}[label = C2.\alph*., ref = C2.\alph*]

			\item
				\label{composition in the first coordinate}
				$H(g f, \id_0) = H(g, \id_0) H(f, \id_0)$,

			\item
				\label{composition in the second coordinate}
				$H(g f, \id_1) = H(g, \id_1) H(g, \id_0)$,

			\item
				\label{morphism in the first coordinate}
				$H(g f, j) = H(g, j) H(f, \id_0)$,

			\item
				\label{morphism in the second coordinate}
				$H(g f, j) = H(g, \id_1) H(f, j)$.

		\end{enumerate}
\end{enumerate*}
For every object~$x$ of~$\cat{C}$ let
\[
	F(x) ≔ H(x, 0) \,,
	\quad
	G(x) ≔ H(x, 1) \,,
\]
and for every morphism~$f \colon x \to y$ in~$\cat{C}$ let
\[
	F(f) ≔ H(f, \id_0) \,,
	\quad
	G(f) ≔ H(f, \id_1) \,,
	\quad
	η_f ≔ H(f, j) \,.
\]
The datum of~$F$ is equivalent to the data~\ref{data of H(x, 0)} and~\ref{data of H(f, id0)}, the datum of~$G$ is equivalent to the data~\ref{data of H(x, 1)} and~\ref{data of H(f, id1)}, and the datum of~$η$ is equivalent to the datum~\ref{data of H(f, j)}.
The combination of conditions~\ref{identities in the first coordinate} and~\ref{composition in the first coordinate} is equivalent to the functoriality of~$F$, the combination of conditions~\ref{identities in the second coordinate} and~\ref{composition in the second coordinate} is equivalent to the functoriality of~$G$.
The combination of conditions~\ref{morphism in the first coordinate} and~\ref{morphism in the second coordinate} is then equivalent to~$η$ defining a natural transformation from~$F$ to~$G$ via~\cref{natural transformations via diagonal morphisms}.

This shows that functors~$H \colon \cat{C} × 𝟚 \to \cat{D}$ correspond to pairs of functors~$F, G \colon \cat{C} \to \cat{D}$ together with a natural transformation~$α \colon F \To G$.
This correspondence is given by~$F = H i_0$ and~$G = H i_1$, and the natural transformation~$α$ corresponds to the morphisms~$H(f, i) \colon F(x) \to G(y)$, where~$f \colon x \to y$ ranges through the morphisms in~$\cat{C}$, as laid out in~\cref{natural transformations via diagonal morphisms}.
