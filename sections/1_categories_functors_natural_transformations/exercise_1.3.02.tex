\subsection{}

Let~$P$ and~$Q$ be two partially ordered sets, and let~$\cat{P}$ and~$\cat{Q}$ be the corresponding categories.
A functor~$F$ from~$\cat{P}$ to~$\cat{Q}$ consists of a function
\[
	f \colon P \to Q
\]
as well as for every two elements~$p_1$ and~$p_2$ of~$P$ a function
\[
	F_{p_1, p_2} \colon \cat{P}(p_1, p_2) \to \cat{Q}(f(p_1), f(p_2))
\]
such that
\begin{equation}
	\label{conditions for functor between preorders}
		F_{p, p}(\id_p) = \id_{f(p)}
		\quad\text{and}\quad
		F_{p_1, p_3}(β α) = F_{p_2, p_3}(β) ∘ F_{p_1, p_2}(α)
\end{equation}
for all~$p, p_1, p_2, p_3 ∈ \cat{P}$ and all morphisms~$α \colon p_1 \to p_2$ and~$β \colon p_2 \to p_3$.

We make two observations regarding the above data and conditions.
\begin{enumerate*}

	\item
		The sets~$\cat{Q}(q_1, q_2)$ for~$q_1, q_2 ∈ Q$ are either empty or singletons.
		The equalities~\eqref{conditions for functor between preorders} are therefore automatically satisfied.

	\item
		The existence of a function~$F_{p_1, 1_2} \colon \cat{P}(p_1, p_2) \to \cat{Q}(f(p_1), f(p_2))$ is equivalent to the implication
		\[
			\cat{P}(p_1, p_2) ≠ ∅ \implies \cat{Q}(f(p_1), f(p_2)) ≠ ∅ \,,
		\]
		which -- by constructions of~$\cat{P}$ and~$\cat{Q}$ -- is furthermore equivalent to the implication
		\[
			p_1 ≤ p_2 \implies f(p_1) ≤ f(p_2) \,.
		\]
		The existence of the functions~$F_{p_1, p_2}$ for all~$p_1, p_2 ∈ P$ is therefore equivalent to~$f$ being isotone (i.e., weakly increasing).

\end{enumerate*}
Together, these two observations show that a functor from~$\cat{P}$ to~$\cat{Q}$ is \enquote{the same} as an isotone function from~$\cat{P}$ to~$\cat{Q}$.
