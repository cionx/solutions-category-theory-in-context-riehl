\subsection{}



\subsubsection{}

Suppose first that~$f$ is a split epimorphism.
This means that there exists a morphism~$g \colon y \to x$ with~$f g = \id_y$.
It follows for every object~$c$ of~$\cat{C}$ for the two induced functions
\[
	f_* \colon \cat{C}(c, x) \to \cat{C}(c, y)
	\quad\text{and}\quad
	g_* \colon \cat{C}(c, y) \to \cat{C}(c, x)
\]
that
\[
	f_*( g_*(φ) )
	=
	f_*(g φ)
	=
	f g φ
	=
	\id_y φ
	=
	φ
\]
for every element~$φ$ of the set~$\cat{C}(c, y)$.
This shows that the function~$g_*$ is right inverse to the function~$f_*$.
The existence of such a right-inverse entails that~$f_*$ is surjective.

Suppose now that the induced function~$f_* \colon \cat{C}(c, x) \to \cat{C}(c, y)$ is surjective for every object~$c$ of~$\cat{C}$.
By choosing~$c$ as~$y$, we can see that the map
\[
	f_*
	\colon
	\cat{C}(y, x) \to \cat{C}(y, y) \,,
	\quad
	g \mapsto f g
\]
is surjective.
This surjectivity entails that there exists an element~$g$ of~$\cat{C}(y, x)$ -- that is, a morphism~$g \colon y \to x$ in~$\cat{C}$ -- such that~$f g = \id_y$.
The existence of this morphism~$g$ tells us that~$f$ is split epimorphism.



\subsubsection{}

We have the following sequence of equivalences:
\begin{align*}
	{}&
	\text{$f \colon x \textto y$ is a split monomorphism in~$\cat{C}$}
	\\
	\iff{}&
	\text{there exists a morphism~$g \colon y \textto x$ in~$\cat{C}$ with~$g f = \id_{x,\cat{C}}$}
	\\
	\iff{}&
	\text{there exists a morphism~$g \colon y \textto x$ in~$\cat{C}$ with~$(g f)^{\op} = \id_{x,\cat{C}}^{\op}$}
	\\
	\iff{}&
	\text{there exists a morphism~$g \colon y \textto x$ in~$\cat{C}$ with~$f^{\op} g^{\op} = \id_{x,\cat{C}^{\op}}$}
	\\
	\iff{}&
	\text{there exists a morphism~$g' \colon x \textto y$ in~$\cat{C}$ with~$f^{\op} g' = \id_{x,\cat{C}^{\op}}$}
	\\
	\iff{}&
	\text{$f^{\op} \colon y \textto x$ is a split epimorphism in~$\cat{C}^{\op}$}
	\\
	\iff{}&
	\text{$(f^{\op})_* \colon \cat{C}^{\op}(c, y) \textto \cat{C}^{\op}(c, x)$ is surjective for every object~$c$ of~$\cat{C}^{\op}$}
	\\
	\iff{}&
	\text{$(f^{\op})_* \colon \cat{C}^{\op}(c, y) \textto \cat{C}^{\op}(c, x)$ is surjective for every object~$c$ of~$\cat{C}$} \,.
\end{align*}
We observe that the diagram
\[
	\begin{tikzcd}
		\cat{C}(y, c)
		\arrow{r}[above]{f^*}
		\arrow{d}[left]{(\ph)^{\op}}
		&
		\cat{C}(x, c)
		\arrow{d}[right]{(\ph)^{\op}}
		\\
		\cat{C}^{\op}(c, y)
		\arrow{r}[above]{(f^{\op})_*}
		&
		\cat{C}^{\op}(c, x)
	\end{tikzcd}
\]
commutes because
\[
	f^*(h)^{\op}
	=
	(h f)^{\op}
	=
	f^{\op} h^{\op}
	=
	(f^{\op})_*( h^{\op} )
\]
for every element~$h$ of~$\cat{C}(y, c)$.
Both vertical arrows in this diagram are bijections.
Consequently, the upper horizontal arrow is surjective if and only if the lower horizontal arrow is surjective.
