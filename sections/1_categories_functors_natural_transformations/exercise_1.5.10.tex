\subsection{}

Let~$\cat{D}$ be a discrete category and let~$\cat{C}$ be a category that is equivalent to~$\cat{D}$.
More explicitly, let~$F \colon \cat{D} \to \cat{C}$ be an equivalence of categories.

The category~$\cat{D}$ is in particular a groupoid.
The category~$\cat{C}$ is therefore also a groupoid, since by Exercise~1.5.iv, the equivalence~$F$ reflect isomorphisms.

We have for every two objects~$d$ and~$d'$ of~$\cat{D}$ the induced bijection
\[
	\cat{D}(d, d') \xto{F} \cat{C}(F d, F d') \,.
\]
We know that the set~$\cat{C}(c, c')$ contains at most one element for every two objects~$c$ and~$c'$ of~$\cat{C}$.
Consequently,~$\cat{D}(d, d')$ consists of at most one element.
This entails that for every object~$d$ of~$\cat{D}$, the automorphism group of~$d$ is trivial.

From these observations and from Exercise~1.5.vii, we altogether find that~$\cat{D}$ is the disjoint union of its isomorphism classes, each of which is equivalent to~$\Base 1$.%
\footnote{We denote the trivial group by~$1$.}
(In terms of graphs, one might picture each isomorphism class as a complete graph.
An edge in this graph represents a mutually inverse pair of non-identity isomorphisms.)
