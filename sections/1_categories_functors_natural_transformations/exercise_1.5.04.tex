\subsection{}



\subsubsection{}

Let~$f \colon x \to y$ be a morphism in~$\cat{C}$ for which the morphism~$F f \colon F x \to F y$ is an isomorphism in~$\cat{D}$.
This means that there exists a morphism~$g \colon F y \to F x$ in~$\cat{D}$ with both~$F f ∘ g = \id_{F y}$ and~$g ∘ F f = \id_{F x}$.

There exists a morphism~$f' \colon y \to x$ in~$\cat{C}$ with~$g = F f'$ because the functor~$F$ is full.
It follows from the sequence of equalities
\[
	F (f ∘ f')
	=
	F f ∘ F f'
	=
	F f ∘ g
	=
	\id_{F y}
	=
	F \id_y
\]
that~$f ∘ f' = \id_y$ because the functor~$F$ is faithful.
We find in the same way that also~$f' ∘ f = \id_x$.
This shows that~$f$ is an isomorphism with inverse~$f'$.

We have thus proven that the inverse of~$F f$ in~$\cat{D}$ lifts uniquely to an inverse of~$f$ in~$\cat{C}$.
This entails that~$f$ is an isomorphism.



\subsubsection{}

That the two objects~$F x$ and~$F y$ in~$\cat{D}$ are isomorphic means that there exists an isomorphism~$g \colon F x \to F y$ in~$\cat{D}$.
There exists a morphism~$f \colon x \to y$ in~$\cat{C}$ with~$g = F f$ because the functor~$F$ is full.
It follows from part~(i) of this exercise that~$f$ is also an isomorphism.
The existence of this isomorphism shows that the two objects~$x$ and~$y$ are isomorphic.
