\subsection{}

Let~$i$ denote the inclusion map from~$ℤ$ to~$ℚ$, which is a homomorphism of rings.
The map~$i$ is injective, and therefore a monomorphism.

Let~$R$ be another ring and let~$f$ be a homomorphism of rings from~$ℚ$ to~$R$.
We have for every nonzero integer~$n$ the equalities
\[
	f(n) ⋅ f\biggl( \frac{1}{n} \biggr)
	= f\biggl( n ⋅\frac{1}{n} \biggr)
	= f(1)
	= 1 \,.
\]
This shows that~$f(1/n)$ is multiplicatively inverse to~$f(n)$ in~$R$.
In other words,~$f(n)$ is invertible in~$R$ and~$f(1 / n) = f(n)^{-1}$.
It follows for every fraction~$p / q$ in~$ℚ$ that
\[
	f\biggl( \frac{p}{q} \biggr)
	= f\biggl( p ⋅ \frac{1}{q} \biggr)
	= f(p) ⋅ f\biggl( \frac{1}{q} \biggr)
	= f(p) ⋅ f(q)^{-1}
	= (f ∘ i)(p) ⋅ (f ∘ i)(q)^{-1} \,.
\]
This shows that the homomorphism~$f$ is uniquely determined by its composite~$f ∘ i$.
As this holds for every morphism~$f$ with domain~$ℚ$, we have shown that~$i$ is an epimorphism.

We have thus shown that~$i$ is both a monomorphism and an epimorphism.
But it is not an isomorphism, as it is not bijective ($1/2$ does not lie in the image of~$i$, whence~$i$ is not surjective).

We can more generally consider a commutative ring~$R$, a multiplicative subset~$S$ of~$R$, and the canonical homomorphism~$j$ from~$R$ to its localization~$S^{-1} R$, given by~$r \mapsto r / 1$.
Then the following hold:
\begin{itemize*}

	\item
		$j$ is always an epimorphism.

	\item
		$j$ is a monomorphism if and only if it is injective, which is the case if and only if~$S$ does not contain any zero divisor.

	\item
		$j$ is an isomorphism if and only if every element of~$S$ is already invertible in~$R$ to begin with.

\end{itemize*}
