\subsection{}

We prove (i) and (ii) by providing two proofs for each statement.



\subsubsection*{Proving (i), first proof}

Let~$h_1, h_2 \colon c \to x$ be two morphisms with~$gf h_1 = gf h_2$.
Then~$f h_1 = f h_2$ because~$g$ is a monomorphism, and then furthermore~$h_1 = h_2$ because~$f$ is a monomorphism.
This shows that the composite~$g f$ is again a monomorphism.



\subsubsection*{Proving (i), second proof}

Let~$c$ be an arbitrary object of~$\cat{C}$.
By assumption, both maps
\[
	f_* \colon \cat{C}(c, x) \to \cat{C}(c, y) \,,
	\quad
	g_* \colon \cat{C}(c, y) \to \cat{C}(c, z)
\]
are injective.
It follows that their composite~$g_* f_*$ is again injective.
But we have the identity~$g_* f_* = (g f)_*$.
We have thus explained why the map
\[
	(g f)_* \colon \cat{C}(c, x) \to \cat{C}(c, z)
\]
is injective for every object~$c$ of~$\cat{C}$.
This then tells us that the composite~$g f$ is a monomorphism.



\subsubsection*{Proving (ii), first proof}

Let~$h_1, h_2 \colon c \to x$ be two morphisms with~$f h_1 = f h_2$.
Then also~$g f h_1 = g f h_2$, and thus~$h_1 = h_2$ because~$g f$ is a monomorphism.
This shows that~$f$ is a monomorphism.



\subsubsection*{Proving (ii), second proof}

Let~$c$ be an arbitrary object of~$\cat{C}$.
By assumption, the map
\[
	(g f)_* \colon \cat{C}(c, x) \to \cat{C}(c, z)
\]
is injective.
But the induced function~$(g f)_*$ equals the composite~$g_* f_*$, whence this composite is injective.
It follows (from naive set theory) that the function~$f_*$ is injective.
We have thus shown that the map
\[
	f_* \colon \cat{C}(c, x) \to \cat{C}(c, y)
\]
is injective for every object~$c$ of~$\cat{C}$.
This tells us that the morphism~$f$ is a monomorphism.



\subsubsection*{Concluding~(i')}

We first observe for every morphism~$f \colon x \to y$ in~$\cat{C}$ the following sequence of equivalences:
\begin{align*}
	{}&
	\text{$f \colon x \textto y$ is an epimorphism in~$\cat{C}$}
	\\
	\iff{}&
	\text{for all~$g_1, g_2 \colon c \textto x$ in~$\cat{C}$,~$g_1 f = g_2 f$ implies~$g_1 = g_2$}
	\\
	\iff{}&
	\text{for all~$g_1, g_2 \colon c \textto x$ in~$\cat{C}$,~$(g_1 f)^{\op} = (g_2 f)^{\op}$ implies~$g_1^{\op} = g_2^{\op}$}
	\\
	\iff{}&
	\text{for all~$g_1, g_2 \colon c \textto x$ in~$\cat{C}$,~$f^{\op} g_1^{\op} = f^{\op} g_2^{\op}$ implies~$g_1^{\op} = g_2^{\op}$}
	\\
	\iff{}&
	\text{for all~$g'_1, g'_2 \colon x \textto c$ in~$\cat{C}^{\op}$,~$f^{\op} g'_1 = f^{\op} g'_2$ implies~$g'_1 = g'_2$}
	\\
	\iff{}&
	\text{$f^{\op} \colon y \textto x$ is a monomorphism in~$\cat{C}^{\op}$} \,.
\end{align*}

Thanks to this observation and part~(i) of the lemma (i.e,~Lemma~1.2.11), we can now observe the following sequence of equivalences:
\begin{align*}
	{}&
	\text{$f \colon x \textto y$ and~$g \colon y \textto z$ are epimorphism in~$\cat{C}$}
	\\
	\iff{}&
	\text{$f^{\op} \colon y \textto x$ and~$g^{\op} \colon z \textto y$ are monomorphisms in~$\cat{C}^{\op}$}
	\\
	\implies{}&
	\text{$f^{\op} g^{\op} \colon z \textto x$ is a monomorphism in~$\cat{C}^{\op}$}
	\\
	\iff{}&
	\text{$(g f)^{\op} \colon z \textto x$ is a monomorphism in~$\cat{C}^{\op}$}
	\\
	\iff{}&
	\text{$g f \colon x \textto z$ is a monomorphism in~$\cat{C}^{\op}$}
\end{align*}
This proves part~(i').



\subsubsection*{Concluding~(ii')}

By using part~(ii) of the lemma and once the previous observation (together with the fact that~$(\ph)^{\op\op} = (\ph)$) we get the following sequence of equivalences:
\begin{align*}
	\SwapAboveDisplaySkip
	{}&
	\text{$gf \colon x \textto z$ is an epimorphism in~$\cat{C}$}
	\\
	\iff{}&
	\text{$(gf)^{\op} \colon z \textto x$ is a monomorphism in~$\cat{C}^{\op}$}
	\\
	\iff{}&
	\text{$f^{\op} g^{\op} \colon z \textto x$ is a monomorphism in~$\cat{C}^{\op}$}
	\\
	\iff{}&
	\text{$g^{\op} \colon z \textto y$ is a monomorphism in~$\cat{C}^{\op}$}
	\\
	\iff{}&
	\text{$g \colon y \textto z$ is an epimorphism in~$\cat{C}$} \,.
\end{align*}
This proves part~(ii').



\subsubsection{Monomorphisms form a subcategory}

A class of morphism~$M$ in~$\cat{C}$ form a subcategory of~$\cat{C}$ if and only if the following conditions are satisfied:
\begin{enumerate*}

	\item
		$M$ is closed under composition.

	\item
		For every morphism~$f \colon x \to y$ belonging to~$M$, both~$\id_x$ and~$\id_y$ again belong to~$M$.

\end{enumerate*}
In the case of a full subcategory (i.e., a subcategory that contains all objects of~$\cat{C}$), the second condition can be simplified as follows:
\begin{enumerate*}

	\item[2’.]
		$\id_x$ belongs to~$M$ for every object~$x$ of~$\cat{C}$.

\end{enumerate*}

We have already seen in Lemma~1.2.11 (and proven in the previous parts of this exercise) that the classes of monomorphisms and epimorphisms are closed under composition.
Identity morphisms are isomorphisms, therefore both split monomorphisms and also split epimorphisms, and therefore both monomorphisms and epimorphisms.
Consequently, both the class of mono\-morphisms and the class of epimorphisms define full subcategories of~$\cat{C}$.
