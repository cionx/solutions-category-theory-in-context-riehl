\subsection{}



\subsubsection*{Definition of the composition of morphisms in~$F ↓ G$}

Let
\[
	(h, k) \colon (d, e, f) \to (d', e', f') \,,
	\quad
	(h', k') \colon (d', e', f') \to (d'', e'', f'')
\]
be two morphisms in~$F ↓ G$ that ought to be composable.
We have by assumptions on the pairs~$(h, k)$ and~$(h', k')$ the following two commutative square diagrams:
\[
	\begin{tikzcd}
		F(d)
		\arrow{r}[above]{f}
		\arrow{d}[left]{F(h)}
		&
		G(e)
		\arrow{d}[right]{G(k)}
		\\
		F(d')
		\arrow{r}[above]{f'}
		&
		G(e')
	\end{tikzcd}
	\qquad
	\begin{tikzcd}
		F(d')
		\arrow{r}[above]{f'}
		\arrow{d}[left]{F(h')}
		&
		G(e')
		\arrow{d}[right]{G(k')}
		\\
		F(d'')
		\arrow{r}[above]{f''}
		&
		G(e'')
	\end{tikzcd}
\]
By combining these two diagrams, with the first diagram above the second, we arrive at the following commutative diagram:
\[
	\begin{tikzcd}
		F(d)
		\arrow{r}[above]{f}
		\arrow{d}[left]{F(h)}
		&
		G(e)
		\arrow{d}[right]{G(k)}
		\\
		F(d')
		\arrow{r}[above]{f'}
		\arrow{d}[left]{F(h')}
		&
		G(e')
		\arrow{d}[right]{G(k')}
		\\
		F(d'')
		\arrow{r}[above]{f''}
		&
		G(e'')
	\end{tikzcd}
\]
By leaving out the center row of this diagram we arrive at the following commutative square diagram:
\[
	\begin{tikzcd}
		F(d)
		\arrow{r}[above]{f}
		\arrow{d}[left]{F(h') F(h)}
		&
		G(e)
		\arrow{d}[right]{G(k') G(k)}
		\\
		F(d'')
		\arrow{r}[above]{f''}
		&
		G(e'')
	\end{tikzcd}
\]
This diagram can equivalently be rewritten as follows:
\[
	\begin{tikzcd}
		F(d)
		\arrow{r}[above]{f}
		\arrow{d}[left]{F(h' h)}
		&
		G(e)
		\arrow{d}[right]{G(k' k)}
		\\
		F(d'')
		\arrow{r}[above]{f''}
		&
		G(e'')
	\end{tikzcd}
\]
The commutativity of this square diagram tells us that the pair~$(h' h, k' k)$ is a morphism from~$(d, e, f)$ to~$(d'', e'', f'')$ in~$F ↓ G$.
We define the composite~$(h', k') ⋅ (h, k)$ as~$(h' h, k' k)$.
In other words, the composition of morphisms in~$F ↓ G$ is componentwise.

The associativity of the composition of morphisms in the proposed category~$F ↓ G$ follows componentwise from the associativity of the compositions of morphisms in~$\cat{D}$ and~$\cat{E}$.

It remains to prove the existence of identity morphisms in~$F ↓ G$.

We have for every object~$(d, e, f)$ in~$F ↓ G$ the following commutative square diagram:
\[
	\begin{tikzcd}
		F(d)
		\arrow{r}[above]{f}
		\arrow{d}[left]{\id_{F(d)}}
		&
		G(e)
		\arrow{d}[right]{\id_{G(e)}}
		\\
		F(d)
		\arrow{r}[above]{f}
		&
		G(e)
	\end{tikzcd}
\]
This diagram can equivalently be rewritten as follows:
\[
	\begin{tikzcd}
		F(d)
		\arrow{r}[above]{f}
		\arrow{d}[left]{F(\id_d)}
		&
		G(e)
		\arrow{d}[right]{G(\id_e)}
		\\
		F(d)
		\arrow{r}[above]{f}
		&
		G(e)
	\end{tikzcd}
\]
The commutativity of this diagram tells us that the pair~$(\id_d, \id_e)$ is a morphism from~$(d, e, f)$ to~$(d, e, f)$ in~$F ↓ G$.
We have for every morphism
\[
	(h, k) \colon (d, e, f) \to (d', e', f')
\]
in~$F ↓ G$ the two sequences of equalities
\[
	(\id_{d'}, \id_{e'}) ⋅ (h, k)
	=
	(\id_{d'} ⋅ h, \id_{e'} ⋅ k)
	=
	(h, k)
\]
and
\[
	(h, k) ⋅ (\id_d, \id_e)
	=
	(h ⋅ \id_d, k ⋅ \id_e)
	=
	(h, k) \,.
\]
This tells us that for every object~$(d, e, f)$ of~$F ↓ G$ the endomorphism~$(\id_d, \id_e)$ of~$(d, e, f)$ serves as the identity morphism of~$(d, e, f)$.

We have altogether constructed a category~$F ↓ G$.



\subsubsection*{The functors~$\dom$ and~$\codom$}

We define the \enquote{domain functor}~$\dom \colon F ↓ G \to \cat{D}$~as~$\dom((d, e, f)) = d$ on objects and as~$\dom((h, k)) = h$ on morphisms.
Similarly, we define the \enquote{codomain functor}~$\codom \colon F ↓ G \to \cat{E}$ as~$\codom((d, e, f)) = e$ on objects and as~$\codom((h, k)) = k$ on morphisms.
The actions of~$\dom$ and~$\codom$ can more graphically be depicted as follows:
\[
	\begin{tikzcd}
		d
		\arrow{d}[left]{h}
		\\
		d'
	\end{tikzcd}
	\qquad\leadsfrom\qquad
	\begin{tikzcd}
		F(d)
		\arrow{r}[above]{f}
		\arrow{d}[left]{F(h)}
		&
		G(e)
		\arrow{d}[right]{G(k)}
		\\
		F(d')
		\arrow{r}[above]{f'}
		&
		G(e')
	\end{tikzcd}
	\qquad\leadsto\qquad
	\begin{tikzcd}
		e
		\arrow{d}[right]{k}
		\\
		e'
	\end{tikzcd}
\]

The assignments~$\dom$ and~$\codom$ are indeed functors because identities in~$F ↓ G$ and composition of morphisms in~$F ↓ G$ work componentwise.
