\subsection{}



\subsubsection*{For isomorphisms}

Every isomorphism of groups~$φ \colon G \to H$ induces isomorphisms of groups
\begin{gather*}
	\Gcenter(φ)
	\colon
	\Gcenter(G) \to \Gcenter(H) \,,
	\quad
	g \mapsto φ(g) \,,
	\\
	\Commutator(φ)
	\colon
	\Commutator(G) \to \Commutator(H) \,,
	\quad
	g \mapsto φ(g) \,,
	\\
	\Aut(φ)
	\colon
	\Aut(G) \to \Aut(H) \,,
	\quad
	ψ \mapsto φ ψ φ^{-1} \,.
\end{gather*}
In this way, the constructions~$\Gcenter$,~$\Commutator$ and~$\Aut$ become functors from~$\Group_{\iso}$ to~$\Group$.



\subsubsection*{For epimorphisms}

We use in following without proof that a homomorphism of groups is an epimorphism in~$\Group$ if and only if it is surjective.

\begin{itemize}

	\item
		Let~$φ \colon G \to H$ be an epimorphism of groups.
		We claim that
		\[
			φ( \Gcenter(G) ) ⊆ \Gcenter(H) \,.
		\]
		Indeed, let~$z$ be an element of~$\Gcenter(G)$ and let~$h$ be an element of~$H$.
		There exists by assumption some element~$g$ of~$G$ with~$h = φ(g)$.
		It follows that
		\[
			φ(z) h = φ(z) φ(g) = φ(z g) = φ(g z) = φ(g) φ(z) = h φ(z) \,.
		\]
		This shows that~$φ(z)$ is contained in~$\Gcenter(H)$, as claimed.

		It follows that the homomorphism~$φ$ restricts to a homomorphism of groups from~$\Gcenter(G)$ to~$\Gcenter(H)$.
		This observation allows us to extend~$\Gcenter$ to a functor from~$\Group_{\epi}$ to~$\Group$.

	\item
		Every homomorphism of groups~$φ \colon G \to H$ induces a homomorphism groups from~$\Commutator(G)$ to~$\Commutator(H)$ by restriction.
		This entails that~$\Commutator$ extends to a functor from~$\Group_{\epi}$ to~$\Group$.

	\item
		An epimorphism of groups~$φ \colon G \to H$ does not necessarily induce a map from~$\Aut(G)$ to~$\Aut(H)$:
		an automorphism~$ψ$ of~$G$ descends to an endomorphism of~$H$ if and only if~$ψ(\ker(φ)) ⊆ \ker(φ)$.

		There nevertheless exists a functor from~$\Group_{\epi}$ to~$\Group$ that assigns to each group its automorphism group.
		This is due to the following observation:

		\begin{claim}[\cite{stackexchange_extending_functor_from_isos_to_epis}]
			\label{extend functors from isomorphisms to epimorphisms}
			Every functor from~$\Group_{\iso}$ to~$\Group$ can be extended to a functor from~$\Group_{\epi}$ to~$\Group$.
		\end{claim}

		This observation in turn relies on the following observation:

		\begin{claim}
			\label{when the composite of epis is an iso}
			Let~$φ \colon G \to H$ and~$ψ \colon H \to K$ be two epimorphisms of groups.
			The composite~$ψ φ$ is an isomorphism if and only if both~$φ$ and~$ψ$ are isomorphisms.
		\end{claim}

		\begin{proof}
			If both~$φ$ and~$ψ$ are isomorphisms then their composite~$ψ φ$ is again an isomorphism.

			Suppose conversely that~$ψ φ$ is an isomorphism.
			This entails that~$ψ φ$ is a monomorphism, whence~$φ$ is a monomorphism.
			As~$φ$ is both a monomorphism and an epimorphism in~$\Group$, it is an isomorphism.
			It follows that~$ψ = ψ φ ⋅ φ^{-1}$ is a composite of isomorphisms and therefore also an isomorphism.
		\end{proof}

		\begin{proof}[Proof of \cref{extend functors from isomorphisms to epimorphisms}]
			Let~$F$ be a functor from~$\Group_{\iso}$ to~$\Group$.
			We define an extension~$F'$ of~$F$ by letting~$F'(φ)$ be the trivial homomorphism from~$F(G)$ to~$F(H)$ for every epimorphism~$φ \colon G \to H$ that is not an isomorphism.
			We need to prove that the assignment~$F'$ is functorial.
			More specifically, we need to check that~$F'$ is compatible with both identities and composition.
			\begin{itemize}

				\item
					Let~$G$ be any group.
					The identity homomorphism~$\id_G$ is an isomorphism, so we have
					\[
						F'(\id_G)
						=
						F(\id_G)
						=
						\id_{F(G)}
						=
						\id_{F'(G)}
					\]
					by the functoriality of~$F$.

				\item
					Let~$φ \colon G \to H$ and~$ψ \colon H \to K$ be two epimorphisms of groups.
					We need to show that~$F'(ψ φ) = F'(ψ) F'(φ)$.
					We distinguish between two cases:
					\begin{casedistinction}

						\item
							Suppose that both~$φ$ and~$ψ$ are isomorphisms.
							Then their composite~$ψ φ$ is again an isomorphism, whence
							\[
								F'(ψ φ)
								=
								F(ψ φ)
								=
								F(ψ) F(φ)
								=
								F'(ψ) F'(φ)
							\]
							by the functoriality of~$F$.

						\item
							Suppose that either~$φ$ or~$ψ$ is not an isomorphism.
							By definition of~$F'$, either~$F'(φ)$ or~$F'(ψ)$ is trivial. The composite~$F'(ψ) F'(φ)$ is therefore again trivial.
							It also follows from \cref{when the composite of epis is an iso} that~$ψ φ$ is not an isomorphism, whence~$F'(ψ φ)$ is trivial.
							This shows that~$F'(ψ φ) = F'(ψ) F'(φ)$, as both sides are trivial with the same domain and same codomain.
						\qedhere

					\end{casedistinction}
			\end{itemize}
		\end{proof}

\end{itemize}



\subsubsection*{For homomorphisms}

\begin{itemize}

	\item
		If~$\Gcenter$ were to extend to a functor from~$\Group$ to~$\Group$, then the commutative diagram
		\[
			\begin{tikzcd}[column sep = normal]
				{}
				&
				\symm_3
				\arrow{dr}[above right]{\sign}
				&
				{}
				\\
				ℤ / 2
				\arrow{ur}
				\arrow{rr}[above]{\id}
				&
				{}
				&
				ℤ / 2
			\end{tikzcd}
		\]
		would result in the following commutative diagram:
		\[
			\begin{tikzcd}[column sep = normal]
				{}
				&
				\Gcenter(\symm_3)
				\arrow{dr}
				&
				{}
				\\
				\Gcenter(ℤ / 2)
				\arrow{ur}
				\arrow{rr}[above]{\id}
				&
				{}
				&
				\Gcenter(ℤ / 2)
			\end{tikzcd}
		\]
		We have~$\Gcenter(ℤ / 2) = ℤ / 2$ and $\Gcenter(\symm_3) = 1$ (the trivial group), and therefore would get the following commutative diagram:
		\[
			\begin{tikzcd}[column sep = normal]
				{}
				&
				1
				\arrow{dr}
				&
				{}
				\\
				ℤ / 2
				\arrow{ur}
				\arrow{rr}[above]{\id}
				&
				{}
				&
				ℤ / 2
			\end{tikzcd}
		\]
		The commutativity of this diagram would mean that the identity homomorphism of~$ℤ / 2$ is trivial.
		But this is not the case!

	\item
		The construction~$\Commutator$ extends to a functor from~$\Group$ to~$\Group$ in the usual way:
		every homomorphism of groups~$φ \colon G \to H$ satisfies~$φ(\Commutator(G)) ⊆ \Commutator(H)$, and therefore restricts to a homomorphism~$\Commutator(φ)$ from~$\Commutator(G)$ to~$\Commutator(H)$.

	\item
		It can be shown that~$\Aut$ cannot be extended to a functor from~$\Group$ to~$\Group$.
		We refer to \cite{stackexchange_aut_cannot_be_extended_functorialy} for more details on this claim.

\end{itemize}
