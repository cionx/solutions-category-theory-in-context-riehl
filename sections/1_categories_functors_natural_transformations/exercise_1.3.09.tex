\subsection{}



\subsubsection*{For isomorphisms}

Every isomorphism of groups~$φ \colon G \to H$ induces isomorphisms of groups
\begin{gather*}
	\Gcenter(φ)
	\colon
	\Gcenter(G) \to \Gcenter(H) \,,
	\quad
	g \mapsto φ(g) \,,
	\\
	\Commutator(φ)
	\colon
	\Commutator(G) \to \Commutator(H) \,,
	\quad
	g \mapsto φ(g) \,,
	\\
	\Aut(φ)
	\colon
	\Aut(G) \to \Aut(H) \,,
	\quad
	ψ \mapsto φ ψ φ^{-1} \,.
\end{gather*}
In this way, the constructions~$\Gcenter$,~$\Commutator$ and~$\Aut$ become functors from~$\Group_{\iso}$ to~$\Group$.



\subsubsection*{For epimorphisms}

We use in following without proof that a homomorphism of groups is an epimorphism in~$\Group$ if and only if it is surjective.

\begin{itemize}

	\item
		Let~$φ \colon G \to H$ be an epimorphism of groups.
		We claim that
		\[
			φ( \Gcenter(G) ) ⊆ \Gcenter(H) \,.
		\]
		Indeed, let~$z$ be an element of~$\Gcenter(G)$ and let~$h$ be an element of~$H$.
		There exists by assumption some element~$g$ of~$G$ with~$h = φ(g)$.
		It follows that
		\[
			φ(z) h = φ(z) φ(g) = φ(z g) = φ(g z) = φ(g) φ(z) = h φ(z) \,.
		\]
		This shows that~$φ(z)$ is contained in~$\Gcenter(H)$, as claimed.

		It follows that the homomorphism~$φ$ restricts to a homomorphism of groups from~$\Gcenter(G)$ to~$\Gcenter(H)$.
		This observation allows us to extend~$\Gcenter$ to a functor from~$\Group_{\epi}$ to~$\Group$.

	\item
		Every homomorphism of groups~$φ \colon G \to H$ induces a homomorphism groups from~$\Commutator(G)$ to~$\Commutator(H)$ by restriction.
		This entails that~$\Commutator$ extends to a functor from~$\Group_{\epi}$ to~$\Group$.

	\item
		An epimorphism of groups~$φ \colon G \to H$ does not necessarily induce a map from~$\Aut(G)$ to~$\Aut(H)$.
		An automorphism~$ψ$ of~$G$ descends to an endomorphism of~$H$ if and only if~$ψ(\ker(φ)) ⊆ \ker(φ)$.
		(However, it is not clear to the author of these answers that there couldn’t exist another functor from~$\Group_{\epi}$ from~$\Group$ that assigns to each group its automorphism group.)

\end{itemize}



\subsubsection*{For homomorphisms}

\begin{itemize}

	\item
		If~$\Gcenter$ were to extend to a functor from~$\Group$ to~$\Group$, then the commutative diagram
		\[
			\begin{tikzcd}[column sep = normal]
				{}
				&
				\symm_3
				\arrow{dr}[above right]{\sign}
				&
				{}
				\\
				ℤ / 2
				\arrow{ur}
				\arrow{rr}[above]{\id_{ℤ / 2}}
				&
				{}
				&
				ℤ / 2
			\end{tikzcd}
		\]
		would result in the following commutative diagram:
		\[
			\begin{tikzcd}[column sep = normal]
				{}
				&
				\Gcenter(\symm_3)
				\arrow{dr}
				&
				{}
				\\
				\Gcenter(ℤ / 2)
				\arrow{ur}
				\arrow{rr}[above]{\Gcenter(\id_{ℤ / 2})}
				&
				{}
				&
				\Gcenter(ℤ / 2)
			\end{tikzcd}
		\]
		We have~$\Gcenter(ℤ / 2) = ℤ / 2$ and $\Gcenter(\symm_3) = 1$ (the trivial group), and therefore would get the following commutative diagram:
		\[
			\begin{tikzcd}[column sep = normal]
				{}
				&
				1
				\arrow{dr}
				&
				{}
				\\
				ℤ / 2
				\arrow{ur}
				\arrow{rr}[above]{\id_{ℤ / 2}}
				&
				{}
				&
				ℤ / 2
			\end{tikzcd}
		\]
		The commutativity of this diagram would mean that the identity morphism of~$ℤ / 2$ is the trivial homomorphism.
		But this is not the case!

	\item
		$\Commutator$ extends to a functor from~$\Group$ to~$\Group$ in the usual way.

	\item
		If~$\Aut$ were to extend to a functor from~$\Group$ to~$\Group$, then it would also extend to a functor from~$\Group_{\epi}$ to~$\Group$.
		This would contradict the previous part of this exercise.

\end{itemize}
