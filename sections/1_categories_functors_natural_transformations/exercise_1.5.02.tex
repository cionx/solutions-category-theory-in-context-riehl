\subsection{}

We will show that the category~$Γ^{\op}$ is isomorphic to~$\Fin^∂$.
As~$\Fin^∂$ is equivalent to~$\Fin_*$, this then also shows that~$Γ^{\op}$ is equivalent to~$\Fin_*$.
(We have seen in Example~1.5.6 that~$\Set^∂$ is equivalent to~$\Set_*$.
This equivalence restricts to an equivalence of categories between~$\Fin^∂$ and~$\Fin_*$.)

To prove the claimed isomorphism between~$Γ^{\op}$ and~$\Fin^∂$, we will construct mutually inverse contravariant functors~$F$ and~$G$ between~$Γ$ and~$\Fin^∂$.
The main observation is that a partially defined function~$f \colon T \to S$ is uniquely determined by its fibres~$f^{-1}(s)$, which are disjoint subsets of~$T$ indexed by~$S$.
Such an indexed family of pairwise disjoint subsets has the same data as a morphism from~$S$ to~$T$ in~$Γ$.



\subsubsection*{The functor~$F$}

We start with the functor~$F \colon Γ^{\op} \to \Fin^∂$.

We define~$F$ on objects as~$F(S) ≔ S$ for every finite set~$S$.

For every morphism~$θ \colon S \to T$ in~$Γ$ let~$F(θ)$ be the partially defined function from~$T$ to~$S$ given by
\[
	F(θ)(t) = s
	\quad\text{if and only if}\quad
	t ∈ θ(s)
\]
for all~$s ∈ S$,~$t ∈ T$.
In other words, the set~$θ(s)$ is the fibre of~$s$ under~$F(θ)$:
\[
	F(θ)^{-1}(s) = θ(s) \,.
\]
The partially defined function~$F(θ)$ is well-defined since the sets~$θ(s)$, where~$s$ ranges through~$S$, are pairwise disjoint.%
\footnote{
	The partially defined function~$F(θ)$ is a total function if and only if the set~$T$ is completely covered by the sets~$θ(s)$ with~$s ∈ S$.
}

We us now verify the functoriality of~$F$.

Let~$S$ be a finite set, regarded as an object of~$Γ$, and let~$ι_S$ be the identity morphism of~$S$ in~$Γ$.
In other words,~$ι_S$ is the function
\[
	ι_S
	\colon
	S \to \Power(S) \,,
	\quad
	s \mapsto \{ s \} \,.
\]
The resulting function~$F(ι_S) \colon S \to S$ is then given by~$F(ι_S)(s) = s$ for every~$s ∈ S$, whence~$F(ι_S)$ is the identity function on~$S$.
This shows that the assignment~$F$ preserves identities.

Let~$θ$ be a morphism in~$Γ$ from a set~$S$ to a set~$T$, and let~$σ$ be a morphism in~$Γ$ from~$T$ to a set~$U$.
(Therefore,~$θ$ and~$σ$ are functions~$θ \colon S \to \Power(T)$ and~$σ \colon T \to \Power(U)$.)
We have for all~$s ∈ S$,~$u ∈ U$ the sequence of equivalences
\begin{align*}
	{}&
	F(σ θ)(u) = s
	\\
	\iff{}&
	u ∈ (σ θ)(s)
	\\
	\iff{}&
	\textstyle u ∈ ⋃_{t ∈ θ(s)} σ(t)
	\\
	\iff{}&
	\text{there exists some~$t ∈ θ(s)$ with~$u ∈ σ(t)$}
	\\
	\iff{}&
	\text{there exists some~$t ∈ T$ with~$t ∈ θ(s)$ and~$u ∈ σ(t)$}
	\\
	\iff{}&
	\text{there exists some~$t ∈ T$ with~$F(θ)(t) = s$ and~$F(σ)(u) = t$}
	\\
	\iff{}&
	F(θ)( F(σ)(u) ) = s
	\\
	\iff{}&
	(F(θ) ∘ F(σ))(u) = s \,.
\end{align*}
This shows that~$F(σ θ) = F(θ) ∘ F(σ)$.

We have thus shown that~$F$ is a contravariant functor from~$Γ$ to~$\Fin^∂$.



\subsubsection*{The functor~$G$}

We now define the functor~$G$.

The action of~$G$ on objects is given by~$G(S) = S$ for every finite set~$S$.

For every partially defined function~$f \colon S \to T$ between finite sets~$S$ and~$T$ let~$G(f)$ be the induced function
\[
	G(f)
	\colon
	T \to \Power(S) \,,
	\quad
	t \mapsto f^{-1}(t) \,.
\]
The fibres~$f^{-1}(t)$, where~$t$ ranges through~$T$, are pairwise disjoint.
The function~$G(f)$ is therefore a morphism from~$T$ to~$S$ in~$Γ$.

We have to verify the contravariant functoriality of~$G$.

Let~$S$ be a set and let~$\id_S$ be the identity function on~$S$.
Then
\[
	G(\id_S)(s) = \id_S^{-1}(s) = \{ s \} = ι_S(s)
\]
for every~$s ∈ S$, where~$ι_S$ denotes the identity morphism of~$S$ in~$Γ$, and therefore~$G(\id_S) = ι_S$.
This shows that the assignment~$G$ preserves identities.

Let~$f \colon S \to T$ and~$g \colon T \to U$ be partially defined functions between sets~$S$,~$T$ and~$U$.
We have for every~$u ∈ U$ the sequence of equalities
\begin{align*}
	G(g ∘ f)(u)
	&=
	(g ∘ f)^{-1}(u)
	\\
	&=
	f^{-1}( g^{-1}(u) )
	\\
	&=
	\textstyle ⋃_{t ∈ g^{-1}(u)} f^{-1}(t)
	\\
	&=
	\textstyle ⋃_{t ∈ G(g)(u)} G(f)(t)
	\\
	&=
	(G(f) ∘ G(g))(u) \,,
\end{align*}
and therefore the equality~$G(g ∘ f) = G(f) ∘ G(g)$.



\subsubsection*{The functors are mutually inverse}

It remains to check that the functors~$F$ and~$G$ are mutually inverse.

We have for every finite set~$S$ the sequences of equalities
\[
	G(F(S)) = G(S) = S \,, \quad
	F(G(S)) = F(S) = S \,,
\]
which tells us that~$F$ and~$G$ are mutually inverse on objects.

Let~$θ$ be a morphism in~$Γ$ from a set~$S$ to a set~$T$.
We have for all~$s ∈ S$,~$t ∈ T$ the sequence of equivalences
\begin{align*}
	\SwapAboveDisplaySkip
	{}&
	s ∈ (GF)(θ)(t)
	\\
	\iff{}&
	s ∈ G(F(θ))(t)
	\\
	\iff{}&
	s ∈ F(θ)^{-1}(t)
	\\
	\iff{}&
	F(θ)(s) = t
	\\
	\iff{}&
	s ∈ θ(t) \,.
\end{align*}
This shows that~$(GF)(θ) = θ$ for every morphism~$θ$ in~$Γ$.

Let~$f \colon S \to T$ be a partially defined function between sets~$S$ and~$T$.
We have for all~$s ∈ S$,~$t ∈ T$ the sequence of equivalences
\begin{align*}
	{}&
	(FG)(f)(s) = t
	\\
	\iff{}&
	F(G(f))(s) = t
	\\
	\iff{}&
	s ∈ G(f)(t)
	\\
	\iff{}&
	s ∈ f^{-1}(t)
	\\
	\iff{}&
	f(s) = t \,.
\end{align*}
This shows that~$(FG)(f) = f$ for every morphism~$f$ in~$\Fin^∂$.

We have thus shown that the two functors~$F$ and~$G$ are mutually inverse.
