\subsection{}

Consider the following diagram of functors and natural transformations:
\begin{equation}
	\label{diagram for horizontal composition of natural transformations}
	\begin{tikzcd}
		\cat{C}
		\arrow[r, bend left = 50, "F", ""{name=U1, below}]
		\arrow[r, bend right = 50, "G", swap, ""{name=D1, above}]
		\arrow[Rightarrow, from=U1, to=D1, "α"]
		&
		\cat{D}
		\arrow[r, bend left = 50, "H", ""{name=U2, below}]
		\arrow[r, bend right = 50, "K", swap, ""{name=D2, above}]
		\arrow[Rightarrow, from=U2, to=D2, "β"]
		&
		\cat{E}
	\end{tikzcd}
\end{equation}
The vertical composite of~$α$ and~$β$ was originally defined componentwise as the composite morphism in the following commutative diagram:
\[
	\begin{tikzcd}[row sep = huge]
		H F c
		\arrow{r}[above]{β_{F c}}
		\arrow[dashed]{dr}[above right]{(β * a)_c}
		\arrow{d}[left]{H α_c}
		&
		K F c
		\arrow{d}[right]{K α_c}
		\\
		H G c
		\arrow{r}[below]{β_{G c}}
		&
		K G c
	\end{tikzcd}
\]
We also know that the horizontal and vertical composition satisfy the interchange rule
\[
	(α ⋅ β) * (γ ⋅ δ) = (α * γ) ⋅ (β * δ)
\]
whenever we are in the following situation:
\[
	\begin{tikzcd}
		\cat{C}
		\arrow[r, bend left = 70, "F", ""{name=A1, below}]
		\arrow[r, "G" description, ""{name=B1, above}, ""{name=C1, below}]
		\arrow[r, bend right = 70, "H", swap, ""{name=D1, above}]
		\arrow[Rightarrow, from=A1, to=B1, "α"]
		\arrow[Rightarrow, from=C1, to=D1, "β"]
		&
		\cat{D}
		\arrow[r, bend left = 70, "J", ""{name=A2, below}]
		\arrow[r, "K" description, ""{name=B2, above}, ""{name=C2, below}]
		\arrow[r, bend right = 70, "L", swap, ""{name=D2, above}]
		\arrow[Rightarrow, from=A2, to=B2, "γ"]
		\arrow[Rightarrow, from=C2, to=D2, "δ"]
		&
		\cat{E}
	\end{tikzcd}
\]
We have also seen that from the horizontal composition of natural transformation we can derive the whiskering of natural transformations:
in the situations
\[
	\begin{tikzcd}
		\cat{C}
		\arrow[r, bend left = 50, "F", ""{name=U1, below}]
		\arrow[r, bend right = 50, "G", swap, ""{name=D1, above}]
		\arrow[Rightarrow, from=U1, to=D1, "α"]
		&
		\cat{D}
		\arrow[r, "H"]
		&
		\cat{E}
	\end{tikzcd}
	\qquad\text{and}\qquad
	\begin{tikzcd}
		\cat{C}
		\arrow[r, "F"]
		&
		\cat{D}
		\arrow[r, bend left = 50, "H", ""{name=U2, below}]
		\arrow[r, bend right = 50, "K", swap, ""{name=D2, above}]
		\arrow[Rightarrow, from=U2, to=D2, "β"]
		&
		\cat{E}
	\end{tikzcd}
\]
we have the induced diagrams
\begin{equation}
	\label{diagrams for the whiskering of natural transformations}
	\begin{tikzcd}
		\cat{C}
		\arrow[r, bend left = 50, "F", ""{name=U1, below}]
		\arrow[r, bend right = 50, "G", swap, ""{name=D1, above}]
		\arrow[Rightarrow, from=U1, to=D1, "α"]
		&
		\cat{D}
		\arrow[r, bend left = 50, "H", ""{name=U2, below}]
		\arrow[r, bend right = 50, "H", swap, ""{name=D2, above}]
		\arrow[Rightarrow, from=U2, to=D2, "\id_H"]
		&
		\cat{E}
	\end{tikzcd}
	\qquad\text{and}\qquad
	\begin{tikzcd}
		\cat{C}
		\arrow[r, bend left = 50, "F", ""{name=U1, below}]
		\arrow[r, bend right = 50, "F", swap, ""{name=D1, above}]
		\arrow[Rightarrow, from=U1, to=D1, "\id_F"]
		&
		\cat{D}
		\arrow[r, bend left = 50, "H", ""{name=U2, below}]
		\arrow[r, bend right = 50, "K", swap, ""{name=D2, above}]
		\arrow[Rightarrow, from=U2, to=D2, "β"]
		&
		\cat{E}
	\end{tikzcd}
\end{equation}
and therefore the natural transformations
\[
	H α ≔ \id_H * α
	\qquad\text{and}\qquad
	β F ≔ β * \id_F \,.
\]

However, we can also conversely express horizontal composition via vertical composition and whiskering:
in the situation of the diagram~\eqref{diagram for horizontal composition of natural transformations} we have the sequence of equalities
\[
	β * α
	=
	(\id_K ⋅ β) * (α ⋅ \id_F)
	=
	(\id_K * α) ⋅ (β * \id_F)
	=
	K α ⋅ β F \,,
\]
as well as the sequence of equalities
\[
	β * α
	=
	(β ⋅ \id_H) * (\id_G ⋅ α)
	=
	(β * \id_G) ⋅ (\id_H * α)
	=
	β G ⋅ H α \,.
\]
These equations give us two ways of expressing horizontal composition via vertical composition and whiskering.

We can derive whiskering from the previous exercise:
In the situations of the diagrams~\eqref{diagrams for the whiskering of natural transformations} we can express the whiskered natural transformations~$H α$ and~$β F$ via the previous exercise as
\[
	H α = H α \Id_{\cat{C}}
	\quad\text{and}\quad
	β F = \Id_{\cat{E}} β F \,.
\]

Altogether we can redefine the horizontal composition of natural transformations via the previous exercise and the vertical composition of natural transformations as
\[
	β * α = \Id_{\cat{E}} K α ⋅ β F \Id_{\cat{C}} \,,
	\quad{\text{or equivalently}}
	β * α = \Id_{\cat{E}} β G ⋅ H α \Id_{\cat{C}} \,.
\]

One upshot of this possible redefinition of the horizontal composition is that the horizontal composite~$β * α$ is indeed a natural transformation.
Indeed, we have already seen that the vertical composite of natural transformations is again a natural transformation, and we have seen in the previous exercise that~$\Id_{\cat{E}} K α$,~$β F \Id_{\cat{C}}$,~$\Id_{\cat{E}} β G$ and~$H α \Id_{\cat{C}}$ are again natural transformations.
