\subsection{}

We denote the category of~\coalgebras{$(\cat{C}, T)$} by~$\CoAlg(\cat{C}, T)$.
Every morphism in~$\CoAlg(\cat{C}, T)$ is also a morphism in~$\cat{C}$.
The composition of morphisms in~$\CoAlg(\cat{C}, T)$ is the composition of morphisms in~$\cat{C}$.
For every coalgebra~$(c, γ)$, the identity morphism of~$(c, γ)$ in~$\CoAlg(\cat{C}, T)$ is the identity morphism of~$c$ in~$\cat{C}$, i.e.,~$\id_{(c, γ)} = \id_c$.

We first observe that every coalgebra~$(c, γ)$ gives rise to another coalgebra, namely~$(T c, T γ)$.
Moreover,~$γ$ is a morphism of coalgebras from~$(c, γ)$ to~$(T c, T γ)$ by the commutativity of the following diagram:
\[
	\begin{tikzcd}
		c
		\arrow{r}[above]{γ}
		\arrow{d}[left]{γ}
		&
		T c
		\arrow{d}[right]{T γ}
		\\
		T c
		\arrow{r}[above]{T γ}
		&
		T^2 c
	\end{tikzcd}
\]

Suppose now that~$(c, γ)$ is a terminal coalgebra, i.e., a terminal object in the category~$\CoAlg(\cat{C}, T)$.
We know that~$γ$ is a morphism of coalgebras from~$(c, γ)$ to~$(T c, T γ)$.
We also know that there exists a (unique) morphism of coalgebras
\[
  δ \colon (T c, T γ) \to (c, γ)
\]
because the coalgebra~$(c, γ)$ is terminal.
This means that~$δ$ is a morphism from~$T c$ to~$c$ in~$\cat{C}$, so that the following diagram commutes:
\begin{equation}
	\label{delta is a morphism of coalgebras}
	\begin{tikzcd}
		T c
		\arrow{r}[above]{δ}
		\arrow{d}[left]{T γ}
		&
		c
		\arrow{d}[right]{γ}
		\\
		T^2 c
		\arrow{r}[above]{T δ}
		&
		T c
	\end{tikzcd}
\end{equation}

The composite~$δ γ$ is a morphism of coalgebras from~$(c, γ)$ to itself.
As~$(c, γ)$ is terminal, this morphism must be the identity morphism of~$(c, γ)$.
Therefore,~$δ γ = \id_{(c, γ)} = \id_c$.
It further follows from the commutativity of the diagram~\eqref{delta is a morphism of coalgebras} that
\[
	γ δ
	=
	T δ ⋅ T γ
	=
	T (δ γ)
	=
	T \id_c
	=
	\id_{T c}
	=
	\id_{(T c, T γ)} \,.
\]
This shows that~$γ$ and~$δ$ are mutually inverse isomorphisms of coalgebras.
