\subsection{}

Let~$\cat{C}$ be the domain of the two functors~$F$ and~$G$.

The naturality of~$α$ ensures for every morphism~$f \colon x \to y$ in~$\cat{C}$ the commutativity of the following square diagram:
\[
	\begin{tikzcd}
		F(x)
		\arrow{r}[above]{F(f)}
		\arrow{d}[left]{α_x}
		&
		F(y)
		\arrow{d}[right]{α_y}
		\\
		G(x)
		\arrow{r}[above]{G(f)}
		&
		G(y)
	\end{tikzcd}
\]
This commutativity is equivalent to the equality~$α_y ∘ F(f) = G(f) ∘ α_x$.
We can rearrange this equality to~$F(f) ∘ α_x^{-1} = α_y^{-1} ∘ G(f)$.
This equality gives us the commutativity of the following diagram:
\[
	\begin{tikzcd}
		G(x)
		\arrow{r}[above]{G(f)}
		\arrow{d}[left]{α_x^{-1}}
		&
		G(y)
		\arrow{d}[right]{α_y^{-1}}
		\\
		F(x)
		\arrow{r}[above]{F(f)}
		&
		F(y)
	\end{tikzcd}
\]
This commutativity tells us that the family~$(α^{-1}_x)_{x ∈ \cat{C}}$ is a natural transformation from~$G$ to~$F$.
