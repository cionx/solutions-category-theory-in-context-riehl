\subsection{}



\subsubsection*{From a bifunctor $\cat{C} × \cat{D} \to \cat{E}$ to functor $\cat{C} \to \cat{E}^{\cat{D}}$}

Let~$F \colon \cat{C} × \cat{D} \to \cat{E}$ be a bifunctor.

We have for every object~$c$ of~$\cat{C}$ the associated inclusion functor
\[
	I_c
	\colon
	\cat{D} \to \cat{C} × \cat{D} \,,
	\quad
	d \mapsto (c, d) \,,
	\quad
	g \mapsto (\id_c, g) \,.
\]
The composite~$F ∘ I_c$ is again a functor.
This composite is precisely~$F(c, \ph)$, whence we have found that~$F(c, \ph)$ is indeed a functor.

To describe the action of~$F(c, \ph)$ on morphisms in~$\cat{C}$, let~$f \colon c \to c'$ be such a morphism.
We have for every morphism~$g \colon d \to d'$ in~$\cat{D}$ the following commutative square diagram in~$\cat{C} × \cat{D}$:
\[
	\begin{tikzcd}[row sep = huge]
		(c, d)
		\arrow{r}[above]{(\id_c, g)}
		\arrow{d}[left]{(f,  \id_d)}
		\arrow[dashed]{dr}[above right]{(f, g)}
		&
		(c, d')
		\arrow{d}[right]{(f, \id_{d'})}
		\\
		(c', d)
		\arrow{r}[below]{(\id_{c'}, g)}
		&
		(c', d')
	\end{tikzcd}
\]
By applying the functor~$F$ to this commutative square diagram, we arrive at the following commutative square diagram in~$\cat{E}$:
\[
	\begin{tikzcd}
		F(c, d)
		\arrow{r}[above]{F(\id_c, g)}
		\arrow{d}[left]{F(f,  \id_d)}
		&
		F(c, d')
		\arrow{d}[right]{F(f, \id_{d'})}
		\\
		F(c', d)
		\arrow{r}[below]{F(\id_{c'}, g)}
		&
		F(c', d')
	\end{tikzcd}
\]
Denoting~$F(f, \id_d)$ as~$F(f, \ph)_d$, we can rewrite this commutative diagram as follows:
\[
	\begin{tikzcd}
		F(c, \ph)(d)
		\arrow{r}[above]{F(c,  \ph)(g)}
		\arrow{d}[left]{F(f, \ph)_d}
		&
		F(c, \ph)(d')
		\arrow{d}[right]{F(f, \ph)_{d'}}
		\\
		F (c', \ph)(d)
		\arrow{r}[below]{F (\id_{c'}, \ph)(g)}
		&
		F (c', \ph)(d')
	\end{tikzcd}
\]
The commutativity of this square diagram tells us that the family
\[
	F(f, \ph) ≔ (F(f, \ph)_d)_{d ∈ \cat{D}}
\]
is a natural transformation from the functor~$F(c, \ph)$ to the functor~$F(c', \ph)$.

It remains to show that~$F(f, \ph)$ is functorial in~$f$.
More precisely, we need to show that
\[
	F(\id_c, \ph) = \id_{F(c, \ph)}
\]
for every object~$c$ of~$\cat{C}$, and that
\[
	F(f' f, \ph) = F(f', \ph) ⋅ F(f, \ph)
\]
for all composable morphisms~$f \colon c \to c'$ and~$f' \colon c' \to c''$ in~$\cat{C}$.
The first equality holds true because
\[
	F(\id_c, \ph)_d
	=
	F (\id_c, \id_d)
	=
	\id_{F (c, d)}
	=
	\id_{F(c, \ph) d}
	=
	(\id_{F(c, \ph)})_d
\]
for every object~$d$ of~$\cat{D}$.
The second equality holds true because
\begin{align*}
	F(f' f, \ph)_d
	&=
	F (f' f, \id_d) \\
	&=
	F (f' f, \id_d \id_d) \\
	&=
	F ( (f', \id_d) (f, \id_d) ) \\
	&=
	F(f', \id_d) F(f, \id_d) \\
	&=
	F(f', \ph)_d F(f, \ph)_d \\
	&=
	(F(f', \ph) ⋅ F(f, \ph))_d
\end{align*}
for every object~$d$ of~$\cat{D}$.


We have overall shown that a bifunctor~$F \colon \cat{C} × \cat{D} \to \cat{E}$ results in a functor~$F' \colon \cat{C} \to \cat{E}^{\cat{D}}$ given on objects by~$F'(c) = F(c, \ph)$ and on morphisms by~$F'(f) = F(f, \ph)$.



\subsubsection*{From a functor~$\cat{C} \to \cat{E}^{\cat{D}}$ to a bifunctor~$\cat{C} × \cat{D} \to \cat{E}$}

Let now~$G$ be a functor from~$\cat{C}$ to~$\cat{E}^{\cat{D}}$.

For every object~$(c, d)$ of~$\cat{C} × \cat{D}$ let
\[
	G' (c, d) ≔ G(c)(d) \,.
\]
We have for every morphism~$(f, g) \colon (c, d) \to (c', d')$ in~$\cat{C} × \cat{D}$ the induced natural transformation~$G(f) \colon G(c) \To G(c')$, and hence the following commutative square diagram:
\[
	\begin{tikzcd}
		G(c)(d)
		\arrow{r}[above]{G(c)(g)}
		\arrow{d}[left]{G(f)_d}
		&
		G(c)(d')
		\arrow{d}[right]{G(f)_{d'}}
		\\
		G(c')(d)
		\arrow{r}[above]{G(c')(g)}
		&
		G(c')(d')
	\end{tikzcd}
\]
We let~$G'(f, g)$ be the diagonal morphism in this diagram, which is a morphism in~$\cat{E}$ from~$G'(c, d)$ to~$G'(c', d')$.

We claim that the assignment~$G'$ is a functor from~$\cat{C} × \cat{D}$ to~$\cat{E}$.
It remains to show the functoriality of~$G'$.
\begin{itemize}

	\item
		We have to show that~$G'(\id_{(c, d)}) = \id_{G'(c, d)}$ for every object~$(c, d)$ in~$\cat{C} × \cat{D}$.
		We have~$\id_{(c, d)} = (\id_c, \id_d)$, whence the morphism~$G'(\id_{(c, d)})$ is defined as the diagonal morphism in the following diagram:
		\[
			\begin{tikzcd}
				G(c)(d)
				\arrow{r}[above]{G(c)(\id_d)}
				\arrow{d}[left]{G(\id_c)_d}
				&
				G(c)(d)
				\arrow{d}[right]{G(\id_c)_{d}}
				\\
				G(c)(d)
				\arrow{r}[above]{G(c)(\id_d)}
				&
				G(c)(d)
			\end{tikzcd}
		\]
		The functor value~$G(c')$ is itself a functor, from~$\cat{D}$ to~$\cat{E}$, whence
		\[
			G(c)(\id_d) = \id_{G(c)(d)} = \id_{G'(c, d)} \,.
		\]
		Similarly, the natural transformation~$G(\id_c)$ equals~$\id_{G(c)}$ by the functoriality of~$G$, whence
		\[
			G(\id_c)_d = (\id_{G(c)})_d = \id_{G(c)(d)} = \id_{G'(c, d)} \,.
		\]
		We can altogether rewrite the above square diagram as follows:
		\[
			\begin{tikzcd}
				G'(c, d)
				\arrow{r}[above]{\id_{G'(c,d)}}
				\arrow{d}[left]{\id_{G'(c,d)}}
				&
				G'(c, d)
				\arrow{d}[right]{\id_{G'(c,d)}}
				\\
				G'(c, d)
				\arrow{r}[above]{\id_{G'(c,d)}}
				&
				G'(c, d)
			\end{tikzcd}
		\]
		We see that the diagonal morphism in this diagram is~$\id_{G'(c, d)}$.

	\item
		We also need to show that~$G'((f', g') (f, g)) = G'(f', g') G'(f, g)$ for every two composable morphisms
		\[
			(f, g) \colon (c, d) \to (c', d')
			\quad\text{and}\quad
			(f', g') \colon (c', d') \to (c'', d'')
		\]
		in~$\cat{C} × \cat{D}$.
		We consider the following commutative diagram:
		\[
			\begin{tikzcd}[column sep = huge, row sep = 5em]
				G(c)(d)
				\arrow{r}[above]{G(c)(g)}
				\arrow{d}[left]{G(f)_d}
				\arrow{dr}[above right]{G'(f, g)}
				&
				G(c)(d')
				\arrow{r}[above]{G(c)(g')}
				\arrow{d}[right]{G(f)_{d'}}
				\arrow{dr}[above right]{G'(f, g')}
				&
				G(c)(d'')
				\arrow{d}[right]{G(f)_{d''}}
				\\
				G(c')(d)
				\arrow{r}[above]{G(c')(g)}
				\arrow{d}[left]{G(f')_d}
				\arrow{dr}[above right]{G'(f', g)}
				&
				G(c')(d')
				\arrow{r}[above]{G(c')(g')}
				\arrow{d}[right]{G(f')_{d'}}
				\arrow{dr}[above right]{G'(f', g')}
				&
				G(c')(d'')
				\arrow{d}[right]{G(f')_{d''}}
				\\
				G(c'')(d)
				\arrow{r}[above]{G(c'')(g)}
				&
				G(c'')(d')
				\arrow{r}[above]{G(c'')(g')}
				&
				G(c'')(d'')
			\end{tikzcd}
		\]
		Leaving out the middle node of this diagram, we get the following commutative diagram:
		\[
			\begin{tikzcd}[row sep = 4em, column sep = 7.2em]
				G(c)(d)
				\arrow{r}[above]{G(c)(g') G(c)(g)}
				\arrow{d}[left]{G(f')_d G(f)_d}
				\arrow{dr}[description]{G'(f', g') G(f, g)}
				&
				G(c)(d'')
				\arrow{d}[right]{G(f')_{d''} G(f)_{d''}}
				\\
				G(c'')(d)
				\arrow{r}[below]{G(c'')(g') G(c'')(g)}
				&
				G(c'')(d'')
			\end{tikzcd}
		\]
	
		It follows from the functoriality of~$G(c)$ that the upper horizontal arrow can be simplified as $G(c)(g' g)$.
		Similarly, the lower horizontal arrow can be simplified as $G(c'')(g' g)$.
		The vertical arrow on the left-hand side can be rewritten as $(G(f') ⋅ G(f))_d$, and thus as~$G(f' f)_d$ by the functoriality of~$G$.
		Similarly, the vertical arrow on the right-hand side can be rewritten as $G(f' f)_{d''}$.
		We get overall the following commutative diagram:
		\[
			\begin{tikzcd}[row sep = 4em, column sep = 7em]
				G(c)(d)
				\arrow{r}[above]{G(c)(g' g)}
				\arrow{d}[left]{G(f' f)_d}
				\arrow{dr}[description]{G'(f', g') G(f, g)}
				&
				G(c)(d'')
				\arrow{d}[right]{G(f' f)_{d''}}
				\\
				G(c'')(d)
				\arrow{r}[below]{G(c'')(g' g)}
				&
				G(c'')(d'')
			\end{tikzcd}
		\]

		The morphism~$G'((f', g')(f, g)) = G'(f' f, g' g)$ is defined as the diagonal morphism in precisely this commutative square diagram.
		Consequently, this morphism agrees with the composite~$G'(f', g') G'(f, g)$.

\end{itemize}



\subsubsection*{The constructions are mutually inverse}

It remains to show that the two constructions are mutually inverse.

Let first~$F$ be a bifunctor from~$\cat{C} × \cat{D}$ to~$\cat{E}$, let~$F'$ be the induced functor from~$\cat{C}$ to~$\cat{E}^{\cat{D}}$, and let~$F''$ be the induced bifunctor from~$\cat{C} × \cat{D}$ to~$\cat{E}$.

We have for every object~$(c, d)$ of~$\cat{C} × \cat{D}$ the sequence of equalities
\[
	F''(c, d)
	=
	F'(c)(d)
	=
	F(c, \ph)(d)
	=
	F(c, d) \,.
\]
For every morphism~$(f, g) \colon (c, d) \to (c', d')$ in~$\cat{C} × \cat{D}$, the morphism~$F''(f, g)$ is defined as the diagonal morphism in the following commutative square diagram:
\[
	\begin{tikzcd}
		F'(c)(d)
		\arrow{r}[above]{F'(c)(g)}
		\arrow{d}[left]{F'(f)_d}
		&
		F'(c)(d')
		\arrow{d}[right]{F'(f)_{d'}}
		\\
		F'(c')(d)
		\arrow{r}[above]{F'(c')(g)}
		&
		F'(c')(d')
	\end{tikzcd}
\]
Using the definition of~$F'$, this diagram can be rewritten as follows:
\[
	\begin{tikzcd}
		F(c, d)
		\arrow{r}[above]{F(\id_c, g)}
		\arrow{d}[left]{F(f, \id_d)}
		&
		F(c, d')
		\arrow{d}[right]{F(f, \id_{d'})}
		\\
		F(c', d)
		\arrow{r}[above]{F(\id_{c'}, g)}
		&
		F(c', d')
	\end{tikzcd}
\]
The diagonal morphism in this diagram is given by
\[
	F(f, \id_{d'}) F(\id_c, g)
	=
	F((f, \id_{d'}) (\id_c, g))
	=
	F(f \id_c, \id_{d'} g)
	=
	F(f, g) \,.
\]
This shows altogether that~$F''(f, g)$ equals~$F(f, g)$.

Let now~$G$ be a functor from~$\cat{C}$ to~$\cat{E}^{\cat{D}}$.
Let~$G'$ be the induced bifunctor from~$\cat{C} × \cat{D}$ to~$\cat{E}$, and let~$G''$ be the induced functor from~$\cat{C}$ to~$\cat{D}^{\cat{E}}$.
We want to show that~$G'' = G$.
To this end we need to show that the two functors~$G''$ and~$G$ agree both on objects and on morphisms.

We first show that~$G''(c) = G(c)$ for every object~$c$ of~$\cat{C}$, i.e., that the two functors~$G$ and~$G''$ agree on objects.
We need to show that the two functors~$G''(c)$ and~$G(c)$ from~$\cat{D}$ to~$\cat{E}$ agree both on objects and on morphisms.
\begin{itemize}

	\item
		Let~$d$ be an arbitrary object of~$\cat{D}$.
		We have the sequence of equalities
		\[
			G''(c)(d) = G'(c, d) = G(c)(d) \,.
		\]
		This tells us that~$G''(c)$ and~$G(c)$ agree on objects.

	\item
		Let~$g \colon d \to d'$ be a morphism in~$\cat{D}$.
		The morphism~$G''(c)(g)$ is defined as~$G'(\id_c, g)$, which in turn is defined as the diagonal morphism in the following commutative diagram:
		\[
			\begin{tikzcd}
				G(c)(d)
				\arrow{r}[above]{G(c)(g)}
				\arrow{d}[left]{G(\id_c)_d}
				&
				G(c)(d')
				\arrow{d}[right]{G(\id_c)_{d'}}
				\\
				G(c)(d)
				\arrow{r}[above]{G(c)(g)}
				&
				G(c)(d')
			\end{tikzcd}
		\]
		We know from the functoriality of~$G$ that
		\[
			G(\id_c)_d
			=
			(\id_{G(c)})_d
			=
			\id_{G(c)(d)}
		\]
		It follows that the diagonal morphism in the above diagram is
		\[
			G(c)(g) G(\id_c)_d
			=
			G(c)(g) \id_{G(c)(d)}
			=
			G(c)(g) \,.
		\]
		This shows that~$G''(c)(g) = G(c)(g)$, so that~$G''(c)$ and~$G(c)$ agree on morphisms.

\end{itemize}
This shows altogether that the two functors~$G''$ and~$G$ agree on objects.

We now show that the two functors~$G''$ and~$G$ agree on morphisms.
To this end, let~$f \colon c \to c'$ be a morphism in~$\cat{C}$.
We need to show that the two natural transformations~$G''(f)$ and~$G(f)$ from~$G''(c) = G(c)$ to~$G''(c') = G(c')$ are equal.
We hence need to show that~$G''(f)_d = G(f)_d$ for every object~$d$ of~$\cat{D}$.
The morphism~$G''(f)_d$ is defined as~$G'(f, \id_d)$, which in turn is defined as the diagonal morphism in the following commutative diagram:
\[
	\begin{tikzcd}
		G(c)(d)
		\arrow{r}[above]{G(c)(\id_d)}
		\arrow{d}[left]{G(f)_d}
		&
		G(c)(d)
		\arrow{d}[right]{G(f)_d}
		\\
		G(c')(d)
		\arrow{r}[above]{G(c')(\id_d)}
		&
		G(c')(d)
	\end{tikzcd}
\]
We know from the functoriality of~$G(c)$ that~$G(c)(\id_d) = \id_{G(c)(d)}$.
The diagonal morphism in the above diagram is therefore given by
\[
	G(f)_d G(c)(\id_d)
	=
	G(f)_d \id_{G(c)(d)}
	=
	G(f)_d \,,
\]
as desired.
