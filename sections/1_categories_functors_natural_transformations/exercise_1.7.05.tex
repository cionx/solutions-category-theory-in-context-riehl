\subsection{}

Let $\cat{C}$ be a category and let~$c$ be an object of~$\cat{C}$.
Let~$\End(c)$ denote the collection of endomorphisms of~$c$ in~$\cat{C}$.
For every two such endomorphisms~$f$ and~$g$, their composite~$gf$ is again an endomorphism of~$\cat{C}$.
The composition of morphisms in~$\cat{C}$ therefore restricts to a binary operation on~$\End(c)$, making~$\End(c)$ into a magma.
As composition of morphism in~$\cat{C}$ is associative, the restricted operation on~$\End(c)$ is again associative, upgrading~$\End(c)$ to a semigroup.
The identity morphism~$\id_c$ acts as a neutral element in~$\End(c)$, further upgrading~$\End(c)$ to a monoid.

Let now~$\cat{C}$ be a category and let~$\cat{D}$ be the functor category~$\cat{C}^{\cat{C}}$.
It follows from the above argumentation that~$\End_{\cat{D}}(\id_c)$ is a monoid under vertical composition of natural transformations.
It follows from the upcoming \lcnamecref{horizontal composition gives monoid structure} that~$\End_{\cat{D}}(\id_c)$ is also a monoid under horizontal composition of natural transformation (with neutral element given by~$\id_{\Id_{\cat{C}}}$, the identity natural transformation of the identity functor on~$\cat{C}$).
These two monoid structures on~$\End_{\cat{D}}(\id_c)$ satisfy the interchange property from Lemma~1.7.7.
It follows from the Eckmann--Hilton argument that both monoid structures agree and are commutative.


\begin{proposition}[Assocativity and units for horizontal composition]
	\label{horizontal composition gives monoid structure}
	\leavevmode
	\begin{enumerate}

		\item
			In the situation
			\[
				\begin{tikzcd}
					\cat{C}
					\arrow[bend left = 40, r, "F", ""{name = U1, below}]
					\arrow[bend right = 40, r, "G", swap, ""{name = D1, above}]
					\arrow[Rightarrow, from = U1, to = D1, "α"]
					&
					\cat{D}
					\arrow[bend left = 40, r, "H", ""{name = U2, below}]
					\arrow[bend right = 40, r, "J", swap, ""{name = D2, above}]
					\arrow[Rightarrow, from = U2, to = D2, "β"]
					&
					\cat{E}
					\arrow[bend left = 40, r, "K", ""{name = U3, below}]
					\arrow[bend right = 40, r, "L", swap, ""{name = D3, above}]
					\arrow[Rightarrow, from = U3, to = D3, "γ"]
					&
					\cat{F}
				\end{tikzcd}
			\]
			we have the equality~$(γ * β) * α = γ * (β * α)$.

		\item
			We have for every natural transformation~$α \colon F \To G$ between two functors~$F, G \colon \cat{C} \to \cat{D}$ the equalities
			\[
				α * \id_{\Id_{\cat{C}}} = α
				\quad\text{and}\quad
				\id_{\Id_{\cat{D}}} * α = α \,.%
				\footnote{
					Note that~$α * \id_{\Id_{\cat{C}}}$ is a natural transformation from~$F \Id_{\cat{C}} = F$ to~$G \Id_{\cat{C}} = G$, and that similarly~$\id_{\Id_{\cat{D}}} * α$ is a natural transformation from~$\Id_{\cat{D}} F = F$ to~$\Id_{\cat{D}} G = G$.
				}
			\]

	\end{enumerate}
\end{proposition}

\begin{proof}
	We prove both claims independent of one another.
	\begin{enumerate}

		\item
			The horizontal composition~$γ * β$ is a natural transformation from~$K H$ to~$L J$, and the horizontal composition~$β * α$ is a natural transformation from~$H F$ to~$J G$.
			We have therefore the sequence of equalities
			\begin{align*}
				(γ * β) * α
				&=
				(γ * β) G ⋅ K H α \\
				&=
				(γ J ⋅ K β) G ⋅ K H α \\
				&=
				γ J G ⋅ K β G ⋅ K H α \\
				&=
				γ J G ⋅ K (β G ⋅ H α) \\
				&=
				γ J G ⋅ K (β * α) \\
				&=
				γ * (β * α) \,.%
				\footnotemark
			\end{align*}
			\footnotetext{
				More generally, given functors~$F_i, G_i \colon \cat{C}_i \to \cat{C}_{i + 1}$ and natural transformations~$α_i \colon F_i \to G_i$ for~$i = 1, \dotsc, n$, we have the identity
				\[
					α_n * \dotsb * α_1
					=
					(α_n G_{n - 1} G_{n - 2} \dotsm G_1)
					⋅ \dotsb
					⋅ (F_n \dotsm F_3 α_2 G_1)
					⋅ (F_n \dotsm F_3 F_2 α_1) \,.
				\]
			}

		\item
			We have the sequences of equalities
			\begin{gather*}
				α * \id_{\Id_{\cat{C}}}
				=
				α \Id_{\cat{C}} ⋅ F \id_{\Id_{\cat{C}}}
				=
				α ⋅ \id_F
				=
				α
			\shortintertext{and}
				\id_{\Id_{\cat{C}}} * α
				=
				\id_{\Id_{\cat{C}}} G ⋅ \Id_{\cat{C}} α
				=
				\id_G ⋅ α
				=
				α \,.
			\end{gather*}
			This shows the claimed equalities.
		\qedhere

	\end{enumerate}
\end{proof}
