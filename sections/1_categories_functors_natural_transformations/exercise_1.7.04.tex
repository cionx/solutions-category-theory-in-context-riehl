\subsection{}

\begin{lemma}[Whiskering and vertical composition]
	\label{whiskering and vertical composition}
	Let~$\cat{C}$,~$\cat{D}$ and~$\cat{E}$ be three categories.
	\begin{enumerate}

		\item
			In the situation
			\[
				\begin{tikzcd}
					\cat{C}
					\arrow[bend left = 70, r, "F", ""{name = U1, below}]
					\arrow[r, "G" description, ""{name = D1, above}, ""{name = U2, below}]
					\arrow[bend right = 70, r, swap, "H", ""{name = D2, above}]
					\arrow[Rightarrow, from = U1, to = D1, "α"]
					\arrow[Rightarrow, from = U2, to = D2, "β"]
					&
					\cat{D}
					\arrow[r, "K"]
					&
					\cat{E}
				\end{tikzcd}
			\]
			we have the equality~$K (β ⋅ α) = K β ⋅ K α$.

		\item
			In the situation
			\[
				\begin{tikzcd}
					\cat{C}
					\arrow[r, "F"]
					&
					\cat{D}
					\arrow[bend left = 70, r, above, "G", ""{name = U1, below}]
					\arrow[r, "H" description, ""{name = D1, above}, ""{name = U2, below}]
					\arrow[bend right = 70, r, swap, "K", ""{name = D2, above}]
					\arrow[Rightarrow, from = U1, to = D1, "α"]
					\arrow[Rightarrow, from = U2, to = D2, "β"]
					&
					\cat{E}
				\end{tikzcd}
			\]
			we have the equality~$(β ⋅ α) F = β F ⋅ α F$.

	\end{enumerate}
\end{lemma}

\begin{proof}
	In the first situation we have the sequence of equalities
	\[
		(K (β ⋅ α))_c
		=
		K (β ⋅ α)_c
		=
		K (β_c α_c)
		=
		(K β_c) (K α_c)
		=
		(K β)_c (K α)_c
		=
		(K β ⋅ K α)_c
	\]
	for every object~$c$ of~$\cat{C}$, and therefore overall the claimed equality of natural transformations~$K (β ⋅ α) = K β ⋅ K α$.
	The second claim can be shown in the same way.
\end{proof}

The diagram
\[
	\begin{tikzcd}[every arrow/.append style={Rightarrow}, row sep = huge]
		J F
		\arrow[bend left = 35]{rr}[above]{(δ ⋅ γ) F}
		\arrow{r}[above]{γ F}
		\arrow{d}[left]{J α}
		\arrow{dr}[above right]{γ * α}
		\arrow[bend right = 40]{dd}[left]{J (β ⋅ α)}
		&
		K F
		\arrow{r}[above]{δ F}
		\arrow{d}[right]{K α}
		\arrow{dr}[above right]{δ * α}
		&
		L F
		\arrow{d}[right]{L α}
		\arrow[bend left = 40]{dd}[right]{L (β ⋅ α)}
		\\
		J G
		\arrow{r}[above]{γ G}
		\arrow{d}[left]{J β}
		\arrow{dr}[above right]{γ * β}
		&
		K G
		\arrow{r}[above]{δ G}
		\arrow{d}[right]{K β}
		\arrow{dr}[above right]{δ * β}
		&
		L G
		\arrow{d}[right]{L β}
		\\
		J H
		\arrow{r}[above]{γ H}
		\arrow[bend right = 35]{rr}[below]{(δ ⋅ γ) H}
		&
		K H
		\arrow{r}[above]{δ H}
		&
		L H
	\end{tikzcd}
\]
commutes by the definition of the horizontal composition of natural transformations (the four inner squares) and by \cref{whiskering and vertical composition} (the four outer parts).
The overall natural transformation from $J F$ to~$L H$ in this diagram is given by $(δ * β) ⋅ (γ * α)$.
But we also know that the diagram
\[
	\begin{tikzcd}[every arrow/.append style={Rightarrow}, column sep = 6em, row sep = huge]
		J F
		\arrow{r}[above]{(δ ⋅ γ) F}
		\arrow{dr}[above right]{(δ ⋅ γ) * (β ⋅ α)}
		\arrow{d}[left]{J (β ⋅ α)}
		&
		L F
		\arrow{d}[right]{L (β ⋅ α)}
		\\
		J H
		\arrow{r}[above]{(δ ⋅ γ) H}
		&
		L H
	\end{tikzcd}
\]
commutes by the definition of the horizontal composition of natural transformations, whence this natural transformation from~$J F$ to~$L H$ is also given by~$(δ ⋅ γ) * (β ⋅ α)$.
Consequently,
\[
	(δ * β) ⋅ (γ * α) = (δ ⋅ γ) * (β ⋅ α) \,.
\]
