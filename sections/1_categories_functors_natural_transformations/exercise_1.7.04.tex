\subsection{}

\begin{lemma}[Whiskering and vertical composition~I]
	\label{two-sided whiskering and vertical composition}
	In the situation
	\[
			\begin{tikzcd}
				\cat{C}
				\arrow{r}[above]{F}
				&
				\cat{D}
				\arrow[bend left = 75, r, "G", ""{name = U1, below}]
				\arrow[r, "H" description, ""{name = D1, above}, ""{name = U2, below}]
				\arrow[bend right = 75, r, swap, "J", ""{name = D2, above}]
				\arrow[Rightarrow, from = U1, to = D1, "α"]
				\arrow[Rightarrow, from = U2, to = D2, "β"]
				&
				\cat{E}
				\arrow{r}[above]{K}
				&
				\cat{F}
			\end{tikzcd}
	\]
	we have the identity
	\[
		K (β ⋅ α) F
		=
		K β F ⋅ K α F \,.
	\]
\end{lemma}

\begin{proof}
	We have for every object~$c$ of~$\cat{C}$ the sequence of equalities
	\begin{align*}
		(K (β ⋅ α) F)_c
		&=
		K (β ⋅ α)_{F c} \\
		&=
		K (β_{F c} ⋅ α_{F c}) \\
		&=
		K β_{F c} ⋅ K α_{F c} \\
		&=
		(K β F)_c ⋅ (K α F)_c \\
		&=
		(K β F ⋅ K α F)_c \,,
	\end{align*}
	and therefore altogether the equality~$K (β ⋅ α) F = K β F ⋅ K α F$.
\end{proof}

\begin{corollary}[Whiskering and vertical composition~II]
	\label{one-sided whiskering and vertical composition}
	Let~$\cat{C}$,~$\cat{D}$ and~$\cat{E}$ be three categories.
	\begin{enumerate}

		\item
			In the situation
			\[
				\begin{tikzcd}
					\cat{C}
					\arrow[bend left = 70, r, "F", ""{name = U1, below}]
					\arrow[r, "G" description, ""{name = D1, above}, ""{name = U2, below}]
					\arrow[bend right = 70, r, swap, "H", ""{name = D2, above}]
					\arrow[Rightarrow, from = U1, to = D1, "α"]
					\arrow[Rightarrow, from = U2, to = D2, "β"]
					&
					\cat{D}
					\arrow[r, "K"]
					&
					\cat{E}
				\end{tikzcd}
			\]
			we have the equality~$K (β ⋅ α) = K β ⋅ K α$.

		\item
			In the situation
			\[
				\begin{tikzcd}
					\cat{C}
					\arrow[r, "F"]
					&
					\cat{D}
					\arrow[bend left = 70, r, above, "G", ""{name = U1, below}]
					\arrow[r, "H" description, ""{name = D1, above}, ""{name = U2, below}]
					\arrow[bend right = 70, r, swap, "K", ""{name = D2, above}]
					\arrow[Rightarrow, from = U1, to = D1, "α"]
					\arrow[Rightarrow, from = U2, to = D2, "β"]
					&
					\cat{E}
				\end{tikzcd}
			\]
			we have the equality~$(β ⋅ α) F = β F ⋅ α F$.
		\qed

	\end{enumerate}
\end{corollary}

\begin{proof}
	\leavevmode
	\begin{enumerate}

		\item
			We have~$K (β ⋅ α) = K (β ⋅ α) \Id_{\cat{C}} = K β \Id_{\cat{C}} ⋅ K α \Id_{\cat{C}} = K β ⋅ K α$.

		\item
			We have~$(β ⋅ α) F = \Id_{\cat{E}} (β ⋅ α) F = \Id_{\cat{E}} β F ⋅ \Id_{\cat{E}} α F = β F ⋅ α F$.
		\qedhere

	\end{enumerate}
\end{proof}

We now return to the given situation:
\[
	\begin{tikzcd}
		\cat{C}
		\arrow[r, bend left = 70, "F", ""{name=A1, below}]
		\arrow[r, "G" description, ""{name=B1, above}, ""{name=C1, below}]
		\arrow[r, bend right = 70, "H", swap, ""{name=D1, above}]
		\arrow[Rightarrow, from=A1, to=B1, "α"]
		\arrow[Rightarrow, from=C1, to=D1, "β"]
		&
		\cat{D}
		\arrow[r, bend left = 70, "J", ""{name=A2, below}]
		\arrow[r, "K" description, ""{name=B2, above}, ""{name=C2, below}]
		\arrow[r, bend right = 70, "L", swap, ""{name=D2, above}]
		\arrow[Rightarrow, from=A2, to=B2, "γ"]
		\arrow[Rightarrow, from=C2, to=D2, "δ"]
		&
		\cat{E}
	\end{tikzcd}
\]
The diagram
\[
	\begin{tikzcd}[every arrow/.append style={Rightarrow}, row sep = huge]
		J F
		\arrow[bend left = 35]{rr}[above]{(δ ⋅ γ) F}
		\arrow{r}[above]{γ F}
		\arrow{d}[left]{J α}
		\arrow{dr}[above right]{γ * α}
		\arrow[bend right = 40]{dd}[left]{J (β ⋅ α)}
		&
		K F
		\arrow{r}[above]{δ F}
		\arrow{d}[right]{K α}
		\arrow{dr}[above right]{δ * α}
		&
		L F
		\arrow{d}[right]{L α}
		\arrow[bend left = 40]{dd}[right]{L (β ⋅ α)}
		\\
		J G
		\arrow{r}[above]{γ G}
		\arrow{d}[left]{J β}
		\arrow{dr}[above right]{γ * β}
		&
		K G
		\arrow{r}[above]{δ G}
		\arrow{d}[right]{K β}
		\arrow{dr}[above right]{δ * β}
		&
		L G
		\arrow{d}[right]{L β}
		\\
		J H
		\arrow{r}[above]{γ H}
		\arrow[bend right = 35]{rr}[below]{(δ ⋅ γ) H}
		&
		K H
		\arrow{r}[above]{δ H}
		&
		L H
	\end{tikzcd}
\]
commutes by definition of the horizontal composition of natural transformations (the four inner squares) and by \cref{one-sided whiskering and vertical composition} (the four outer parts).
The above diagram has the following subdiagram:
\[
	\begin{tikzcd}[
			every arrow/.append style = {Rightarrow},
			column sep = 7em,
			row sep = 4em
		]
		J F
		\arrow{r}[above]{(δ ⋅ γ) F}
		\arrow{dr}[above right]{(δ * β) ⋅ (γ * α)}
		\arrow{d}[left]{J (β ⋅ α)}
		&
		L F
		\arrow{d}[right]{L (β ⋅ α)}
		\\
		J H
		\arrow{r}[below]{(δ ⋅ γ) H}
		&
		L H
	\end{tikzcd}
\]
But the composite of the upper horizontal arrow and right-side vertical arrow is given by
\[
	L(β ⋅ α) ⋅ (δ ⋅ γ)F
	=
	(δ ⋅ γ) * (β ⋅ α) \,.
\]
(The same goes for the composite of the left-side vertical arrow with the lower horizontal arrow.)
Consequently,
\[
	(δ * β) ⋅ (γ * α) = (δ ⋅ γ) * (β ⋅ α) \,.
\]
