\subsection{}



\subsubsection*{The forgetful functor~$\Ab \to \Group$}

Let~$F$ be the inclusion functor from~$\Ab$ to~$\Group$.

The functor~$F$ is full and faithful because~$\Ab$ is a full subcategory of~$\Group$.
The functor~$F$ is not essentially surjective because no non-abelian group is isomorphic to an abelian group.
(And non-abelian groups do, in fact, exist.)

We see in particular that~$F$ is not an equivalence of categories.



\subsubsection*{The forgetful functor~$\Ring \to \Ab$}

Let~$F$ be the forgetful functor from~$\Ring$ to~$\Ab$.

The functor~$F$ is faithful.
But it is not full because not every additive map between rings is a homomorphism of rings, as not every additive map between rings is multiplicative.

The functor~$F$ is not essentially surjective because not every abelian group can be endowed with the structure of a unitary ring.
Consider, for example, the abelian group~$A ≔ ℚ / ℤ$.
A ring structure on~$A$ would consist of a multiplication map
\[
	A ⊗_ℤ A \to A \,,
	\quad
	a ⊗ b \mapsto ab \,.
\]
But the group~$A ⊗_ℤ A$ is trivial.
The only~\bilinear{$ℤ$} multiplication on~$A$ is therefore the zero multiplication, which is non-unital.

We find in particular that the functor~$F$ is not an equivalence of categories.



\subsubsection*{The functor~$(\ph)^× \colon \Ring \to \Group$}

Let~$F$ be the functor~$(\ph)^× \colon \Ring \to \Group$ that assigns to each ring its group of invertible elements (also known as its group of units).

The functor~$F$ is not faithful.
To see this, let~$R$ be an integral domain, let~$a$ and~$b$ be two distinct elements of~$R$, and let~$φ, ψ \colon R[x] \to R$ be the two evaluation homomorphisms determined by~$φ(x) = a$ and~$ψ(x) = b$.
The group~$R[x]^×$ is simply~$R^×$ because~$R$ is an integral domain, and both~$φ$ and~$ψ$ induce the identity homomorphism on~$R^×$.

The functor~$F$ is also not full:
while there exists no homomorphism of rings from~$𝔽_3$ to~$𝔽_5$, there nevertheless exists a homomorphism of groups from~$𝔽_3^× ≅ ℤ/2$ to~$𝔽_5^× ≅ ℤ/4$;
even a non-trivial one.

Regarding the essential surjectivity of~$F$, we need to examine if every group can occur as the group of units of some ring.
This is not the case, as explained in \autocite{stackexchange_groups_of_units_of_rings}.

The functor~$F$ is in particular not an equivalence of categories.



\subsubsection*{The forgetful functor~$\Ring \to \Rng$}

Let~$F$ be the forgetful functor from~$\Ring$ to~$\Rng$.

The functor~$F$ is faithful because~$\Ring$ is a subcategory of~$\Rng$.
The functor~$F$ is not full because there exists a homomorphism from the zero ring to~$ℤ$ in~$\Rng$, but not in~$\Ring$.

The functor~$F$ is also not essentially surjective, because no non-unitary ring is isomorphic to a unitary ring.



\subsubsection*{The forgetful functor~$\Field \to \Ring$}

Let~$F$ be the forgetful functor from~$\Field$ to~$\Ring$.

The functor~$F$ is full and faithful because~$\Field$ is a full subcategory of~$\Ring$.
The functor~$F$ is not essentially surjective because every non-field ring, e.g.,~$ℤ$, is not isomorphic to a field.



\subsubsection*{The forgetful functor~$\Mod_R \to \Ab$}

Let~$F$ be the forgetful functor from~$\Mod_R$ to~$\Ab$.

The functor~$F$ is faithful for every ring~$R$.
Whether the functor~$F$ is full depends on the ring~$R$:
\begin{itemize*}

	\item
		The functor~$F$ will typically not be full, because an additive map between~\modules{$R$} is not necessarily~\linear{$R$}.

	\item
		If~$R$ is either a quotient or localization of~$ℤ$, then every additive map between~$R$-modules is already~\linear{$R$}.

	\item
		Suppose more generally that the unique homomorphism of rings from~$ℤ$ to~$R$ is an epimorphism.

		It then follows that the two canonical homomorphisms of rings from~$R$ to~$R ⊗_ℤ R$ are equal, since they are equal after pre-composition with the homomorphism~$ℤ \to R$.
		In other words,~$r ⊗ 1 = 1 ⊗ r$ in~$R ⊗_ℤ R$ for every~$r ∈ R$.

		Let now~$M$ and~$N$ be two~\modules{$R$} and let~$f \colon M \to N$ be an additive map.
		For~$m ∈ M$ we can then consider the auxiliary map
		\[
			h
			\colon
			R ⊗_ℤ R \to N \,,
			\quad
			r_1 ⊗ r_2 \mapsto r_1 f (r_2 m)
		\]
		because~$f$ is additive.
		We find that
		\[
			r f(m)
			=
			h(r ⊗ 1)
			=
			h(1 ⊗ r)
			=
			f(r m)
		\]
		for every~$r ∈ R$.
		This shows that the map~$f$ is already~\linear{$R$}.

		Our argumentation is essentially taken from \cite{stackexchange_epimorphisms_induces_full_forgetful_functor}.
		A ring for which the unique homomorphism of rings~$ℤ \to R$ is an epimorphism is called \defemph{solid}.
		More information and references about solid rings can be found at \cite{stackexchange_solid_rings}.

\end{itemize*}

The functor~$F$ is essentially surjective if and only if each abelian group can be endowed with an~\module{$R$} structure.
This is the case if and only if the unique homomorphism rings from~$ℤ$ to~$R$ splits, i.e., if and only if there exists a homomorphism of rings from~$R$ to~$ℤ$:
\begin{itemize*}

	\item
		Suppose that such a homomorphism of rings~$φ \colon R \to ℤ$ exists.
		For every abelian group~$A$ we can then pull back the unique~\module{$ℤ$} structure on~$A$ along~$φ$ to an~\module{$R$} structure on~$A$.

	\item
		Suppose conversely that every abelian group~$A$ can be endowed with the structure of an~\module{$R$}.
		This means that we have for every abelian group~$A$ a homomorphism of rings from~$R$ to~$\End_ℤ(A)$.
		For~$A = ℤ$ we have~$\End_ℤ(A) = ℤ$, and therefore a homomorphism of rings from~$R$ to~$ℤ$.

\end{itemize*}

Moreover, there exists a homomorphism of rings from~$R$ to~$ℤ$ if and only if the ring~$R$ is of the form~$R ≅ R' ⋊ ℤ$ for some possibly non-unitary ring~$R'$.
That is, if and only if~$R$ is the unitalization of~$R'$.

The functor~$F$ typically won’t be an equivalence.
